\section{Prozessor Auslastung}
\label{sec:cpulast}
Um die Annahme, die am Ende von \cref{sec:verschickteDaten} gemacht wurde zu "uberpr"ufen, dass %
ein hohe RTT einer hohen Auslastung der CPU des Zielsystems entpricht wird nun der Leerlauf den %
die CPUs haben betrachtet. Da man "uber den Leerlauf r"uckschl"usse "uber die Auslastung der CPU machen kann %
Die Theorie die durch die RTT von Host Lion aufgestellt wurde, dass dieser aufgrund der hohen RTT %
unter der h"ochsten Last stand best"atigt sich wenn man sich die \cref{tab:compCPU} ansieht. %
Vergleicht man den Leerlauf von Lion mit denen der anderen Hosts, ist der Host Lion derjenige mit %
dem geringsten Leerlauf. Im 2000 Megabyte Testfall jedoch ist die CPU vom Host Tusk am meisten %
ausgelastet. Der Leerlauf beim Host Tusk liegt mit 12,05 \% unter der h"alfte der anderen Hosts, im Test. %
Dies ist auch der einzige Test in dem der Host Lion nicht die h"ochste Auslastung hat. Die hohe Auslastung %
des Hosts Tusk h"angt, aber auch mit dem Fehler zusammen der w"ahrend des Tests aufgetreten ist und in dem %
\cref{sec:2000MBTest,sec:fehler} schon aufgef"uhrt ist. %

Wie man auch direkt sehen kann sind die drei von den vier Hosts im 2000 Megabyte Test mit einem gro"sen Abstand %
unausgelastet. Mit jeweils 29 \% haben diese Hosts, die geringste CPU-Last im vergleich zu dem 20 Megabyte und %
200 Megabyte Test. Jedoch ist Tusk mit 12,05 \% der ausgelasteteste Host in diesem. Dies wirkt sich auch auf die %
durchschnittliche Last im Test wieder, mit 23,67 \% liegt diesmal der 2000 Megabyte Test im Mittelfeld zwischen %
dem 20 Megabyte und 200 Megabyte Test. Bei 20 Megabyte Paketen sind Hosts im Test mit 23,83 \% die mit dem %
h"ochsten Leerlaufprozess. W"ahrend beim 200 Megabyte Test mit \mbox{23,35 \%} der geringste Leerlaufprozess erreicht %
ist. Aus der Standarbweichung der einzelnen Tests kann man auch Schlussfolgern, welcher der Tests eine m"oglichst %
gleichm"a"sige Belastung der Hosts hat. So ist werden beim 20 Megabyte Test, alle Hosts mit einer Standardabweichung %
von $\pm$ 0,10 \% am gleichm"a"sigsten belastet. Gefolgt vom 200 Megabyte Test, bei diesem Test ist auch noch eine %
geringe Standardabweichung gegeben mit $\pm$ 0,98 \% sind auch dort die Hosts, gleichm"a"sig belastet. %
Anders als beim 2000 Megabyte Test, die CPU Lastenverteilung ist sehr ungleich, mit $\pm$ 7,05 \% ist die Lastverteilung %
am ungleichm"a"sigsten der Host Tusk hat in diesem Test die h"ochste Last von allen Host. % 


\begin{table}
\centering
\begin{tabular}{l%
 r<{\,\%}%
 r<{\,\%}%
 r<{\,\%}%
}
Agent  				& \multicolumn{1}{r}{20 Megabyte Idle}			& \multicolumn{1}{r}{200 Megabyte Idle}			& \multicolumn{1}{r}{2000 Megabyte Idle}		\\
\hline
Dazzle 				& 25,45			 				& 21,83							& 29,48					\\
Tusk 				& 23,26							& 25,57							& 12,05					\\
Tinker				& 23,78							& 24,55							& 29,16					\\
Lion				& 22,82							& 21,48							& 29,16					\\ 
Agent $\diameter $  		& 23,83							& 23,35					 		& 23,67					\\   
Agent $\sigma $			& 0,10		 					& 0,98							& 7,05      				\\
\end{tabular}
\caption{Leerlauf der CPUs im vergleich. Werte aus den \cref{tab:CPUlastverteilung20Mb,tab:CPUlastverteilung200Mb,tab:CPUlastverteilung2000Mb}}
\label{tab:compCPU}
\end{table}

