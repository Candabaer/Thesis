\section{20 Megabyte Testlauf}
\label{subsec:20MBTest}
%Geringe Standardabweichung gut weil ? Begr"undung 
Der erste durchgef"uhrte Test basiert auf 20 Megabyte Paketen. Dazu wurde das Hintergrundrauschen %
so eingestellt, dass die Gr"o"se der verschickten Pakete 20 Megabyte betr"agt. 

Wie man in \cref{fig:Eth0DazzleStandard} sehen kann, ist der durchschnittliche ausgehende Traffic auf dem Pi Dazzle 8,85 Mbit/s. %
Der eingehende Traffic betr"agt durchschnittlich 8,13 Mbit/s. In der \cref{tab:EingehenderTraffic20Mb} %
ist der Datenverkehr der einzelnen Hosts aufgelistet. Betrachtet wird der minimale, maximale und der durchschnittlich eingehende Traffic. %
Der maximal eingehende Datenstrom betr"agt durchschnittlich 19,205 Mbit/s. Der minimal eingehende Traffic liegt bei %
durchschnittlich 28,45 Kbit/s. Die minimalen Werte entstehen, wenn der Pi keine Pakete empf"angt und nur noch mit dem Zabbix Server kommuniziert. %
Dies kann in den folgenden Tests auch so beobachtet werden. Die Standardabweichung der durchschnittlichen %
und der maximalen Last sind geringer als 1 Mbit/s. Aus dieser geringen Abweichung vom Durchschnitt kann man schlie"sen, dass die Hosts %
gleichm"a"sig ausgelastet sind. % 

\begin{table}
\centering
\begin{tabular}{l%
 r<{\,Kb/s}%
 r<{\,Mb/s}%
 r<{\,Mb/s}%
}
Agent  				& Minimal		& Durchschnitt		& Maximal	\\
\hline
Dazzle 				& 31,64			& 8,13 			& 18,54		\\
Tusk 				& 29,11			& 8,51			& 19,15		\\
Tinker				& 34,3			& 8,33			& 20,28		\\
Lion				& 18,76			& 8,4			& 18,85		\\ 
Agent $\diameter $	 	& 28,4525		& 8,3425		& 19,205	\\   
Agents $\sigma $		& 5,88919084		& 0,138451255		& 0,6570578361	\\

\end{tabular}
\caption{Eingehender Traffic auf den Ethernet Ports bei 20 Megabyte Paketen auf allen Pis.}
\label{tab:EingehenderTraffic20Mb}
\begin{tabular}{l%
 r<{\,Kb/s}%
 r<{\,Mb/s}%
 r<{\,Mb/s}%
}
Agent  				& Minimal		& Durchschnitt		& Maximal	\\	
\hline
Dazzle 				& 628,48		& 8,85	 		& 15,02		\\
Tusk 				& 61,18			& 8,47			& 15,35		\\
Tinker				& 421,76		& 8,52			& 13,95		\\
Lion				& 3030			& 8,38			& 13,94		\\ 
Agent $\diameter $	 	& 1035,35		& 8,5555		& 14,565	\\   
Agents $\sigma $		& 1169,3664626947	& 0,177552809		& 0,6308922253	\\

\end{tabular}
\caption{Ausgehender Traffic auf den Ethernet Ports bei 20 MB Paketen auf allen Pis.}
\label{tab:AusgehenderTraffic20Mb}

\end{table}

Da aber alle Hosts nicht nur Empf"anger, sondern auch gleichzeitig Sender sind, wird auch der ausgehende %
Datenstrom betrachtet. Aus der \cref{tab:AusgehenderTraffic20Mb} kann man ablesen, dass durchschnittlich %
8,56 Mbit/s von jedem einzelnen Host verschickt werden. Wie man sieht, betr"agt die Standardabweichung %
$\sigma$ vom ausgehenden Datenstrom 0,18 Mbit/s. Auch die Maximallast hat nur eine geringe Standardabweichung. %
Die Werte der Maximallast und des Durchschnitts "ahneln sehr der vom eingehenden Traffic. %
Der Wert der Minimallast des Hosts Lion weicht mit 3,03 Mbit/s jedoch deutlich ab, wodurch sich der %
Durchschnitt der Minimallast drastisch erh"oht. %
Man kann auch beobachtenn, dass der Durchschnitt des maxmimal ausgehenden Traffics \mbox{4 Mbit/s} %
geringer ist als der des eingehenden. %

\begin{table}
\centering
\begin{tabular}{l%
 r<{\,KB/s}%
 r<{\,KB/s}%
 r<{\,Ops/s}%
}
Agent	  			& Schreiben	 	& Lesen			& Input/Output 		\\	
\hline
Dazzle 				& 1,04			& 0			& 153,09	        \\
Tusk 				& 1,11			& 0			& 42,4			\\
Tinker				& 1,07 			& 0			& 142,5		 	\\
Lion				& 1,04			& 0			& 144,31	 	\\
Agent $\diameter $  		& 1,065			& 0			& 120,725		\\   
Agent $\sigma $ 		& 0,028722813  		& 0			& 45,3117		\\
\end{tabular}
\caption{I/O Zeiten bei Normalbetrieb auf den Pis}
\label{tab:NormalbetriebIoStat20Mb}
\end{table}


In der \cref{fig:IoStatDazzleStandard} sieht man, dass die Festplatte des Raspberry Pis konstant beschrieben wird. %
Leseoperationen finden "uberhaupt nicht statt. %
Im Durchschnitt werden 1,07 Kilobytes pro Sekunde geschrieben, %
was nicht ann"ahrend der Maximalen Schreibgeschwindigkeit der verwendeten %
SanDisk SD Karten entspricht, welche bei 30 Megabyte pro Sekunde liegt \autocite{san:sd}. %
Aus der \cref{tab:NormalbetriebIoStat20Mb} l"asst sich also schlie"sen, dass die Festplatten der Agents kaum belastet werden. % 

\begin{table}
\centering
\begin{tabular}{l%
 r<{\,\%}%
 r<{\,\%}%
 r<{\,\%}%
 r<{\,\%}%
 r<{\,\%}%
}
Agent  				& Idle			& User Time		& System Time		& I/O wait Time	& Software IRQ	\\
\hline
Dazzle 				& 25,45			& 37,71			& 22,32 		& 1,42		& 16,10	\\
Tusk 				& 23,26			& 37,41			& 23,57			& 0,22		& 15,54	\\
Tinker				& 23,78			& 35,99			& 22,80			& 1,65		& 15,77	\\
Lion				& 22,82			& 35,99			& 23,15			& 1,73		& 15,84	\\ 
Agent $\diameter $  		& 23,83			& 36,14			& 22,96			& 1,26	 	& 15,81	\\   
Agent $\sigma $			&  0,10			&  0,97			&  0,46			& 0,61		& 0,20  \\ %
%
\end{tabular}
\caption{CPU Last Verteilung}
\label{tab:CPUlastverteilung20Mb}
\end{table}

Betrachtet man nun die \cref{tab:CPUlastverteilung20Mb} spiegeln sich die Ergebnisse aus der \cref{tab:NormalbetriebIoStat20Mb} wider. %
Die Zeit die, die CPU braucht um auf den Abschluss einer Lese-/Schreib Operation zu warten, h"alt sich sehr gering. %
So werden durchschnittlich 1,26 \% der CPU Zeit f"ur Lese-/Schreib Operationen verwendet. Der Gro"steil der CPU Zeit wird %
bei allen Hosts daf"ur verwendet, die Useranwendungen laufen zu lassen. Das entspricht im Schnitt 36,14 \%. Die Idle Spalte zeigt an wieviel Prozent der Prozessorleistung %
ohne eine Aufgabe war. W"ahrend die Systemzeit f"ur die Verwaltung von Netzwerkaufgaben, wie dem Versenden und Empfangen von %
im Netzwerk verschickten Paketen, zust"andig war. %\emph{An Julian: F"ur was ist die Software IRQ time ? Durch was werden die ganzen Software Interrupt %
%ausgel"ost ich peils nicht.}%
Wenn man nun die Werte der Standardabweichung der CPU betrachtet, best"atigt sich die aus der% 
\cref{tab:EingehenderTraffic20Mb} und der \cref{tab:AusgehenderTraffic20Mb} gewonnene Annahme, dass n"amlich das alle Hosts gleichm"a"sig belastet worden sind. %
Die Standardbweichung der Erwartungswerte "ubersteigt 1 \% nicht, was ein Indikator daf"ur ist, dass die %
Prozessoren auf den Hosts in etwa dieselbe Last tragen.  

Es wurde "uber eine Dauer von 12 Stunden 10937 Pakete im Netzwerk verschickt. Die Menge der versendeten Daten in diesem Netzwerk betr"agt 213,61 Gigabyte. %
Rechnet man dies auf eine Stunde herunter, erh"alt man 17,8 Gigabyte pro Stunde oder 911,41 Pakete pro Stunde. Wieder l"asst sich eine ausgewogene Verteilung erkennen. %
Die Standardabweichung der erfolgreich verschickten Pakete liegt bei 15 Paketen und auch die Fehlerrate ist sehr gering. Ein Host %
hat sogar innerhalb von 12 Stunden kein einziges Paket erfolglos absenden k"onnen. Generell ist eine sehr geringe Menge an Daten verloren %
gegangen. Es haben Insgesamt 80 Megabyte ihr Ziel nicht ordnungsgem"a"s erreicht. %  
\begin{table}
\centering
\begin{tabular}{l%
 r<{\,}%
 r<{\,}%
 r<{\,\%}%
 r<{\,GB}%
}
Agent  				& Erfolgreich gesendet			& Erfolglos gesendet			& Erfolglos gesendet	& Verschickte	\\
\hline
Dazzle 				& 2731			 		& 1					& 0,04			& 53,34			\\
Tusk 				& 2710					& 0					& 0,00			& 52,93			\\
Tinker				& 2751					& 2					& 0,07			& 53,73			\\
Lion				& 2745					& 1					& 0,04			& 53,61			\\ 
Summe				& 10923					& 4					& 0,03 			& 213,61		\\
Agent $\diameter $  		& 2734,25				& 1				 	& 0,0375		& 53,40 		\\   
Agent $\sigma $			& 15,76983	 			& 0,70711				& 0,01427      		& 0,308004		\\
\end{tabular}
\caption{Anzahl der gesendeten Pakete "uber einen Zeitraum von 12 Stunden.}
\label{tab:VerschickteDaten20Mb}
\end{table}


In der \cref{tab:VerschickteDaten20Mb} ist die Anzahl der verschickten und empfangenen Pakete aufgelistet %
in den Zeilen stehen die sendenden Hosts, in den Spalten die empfangenden Hosts, sprich: Dazzle schickt %
an Tusk 685 Pakete w"ahrend dem Gesamten Testlauf oder auch. Tinker empf"angt von Lion 679 Pakete. %  
Die Tabellen in \cref{sec:200MBTest} und \cref{sec:2GBTest} lesen sich gleich. % 


\begin{table}
\centering
\begin{tabular}{l%
 r<{\,}%
 r<{\,}%
 r<{\,}%
 r<{\,}%
 r<{\,}%
}
Agent  				& Dazzle	& Tusk		& Tinker	& Lion		& Summe		\\
\hline
Dazzle 				& 659		& 685		& 678		& 707		& 2719		\\
Tusk 				& 668		& 701		& 678		& 661		& 2708		\\
Tinker				& 692		& 659		& 659		& 724		& 2752		\\
Lion				& 703		& 671		& 679		& 691		& 2744		\\ 
Summe				& 2712		& 2716		& 2694 		& 2694		& 10923		\\
\end{tabular}
\caption{Die Anzahl der verschickten und der empfangen 20 Megabyte Pakete pro Host.}
\label{tab:VerschicktePakete20Mb}
\end{table}



























