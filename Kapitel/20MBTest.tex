\subsection{20 Megabyte Testlauf}
\label{subsec:20MBTest}
Der erste durchgef"uhrte Test basiert auf 20 Megabyte Paketen, dazu wurde das Hintergrundrauschen %
so eingestellt das die gr"o"se der verschickten Pakete 20 Megabyte betr"agt. 

Wie man in \cref{fig:Eth0DazzleStandard} sehen kann ist der durchschnittlich eingehende Durchsatz auf dem Pi Dazzle 8,04 Megabit %
pro Sekunde. Der ausgehende Traffic betr"agt durchschnittlich 8,62 Megabit pro Sekunde. Rechnet man diese Werte in Bytes pro Sekunde um %
betr"agt der ausgehende Datenstrom 1,0775 Megabyte/s und der durchschnittlich eingehende Datenstrom 1,005 Megabyte/s %
Daraus k"onnen wir schlie"sen das der Ethernet Port von Dazzle unter einer durschnittlichen I/O-Last von 2,0825 Megabyte/s stand. % 
Woraus folgt das der Ethernet Port zu 0,20825\% belastet ist.
\begin{table}
\centering
\begin{tabular}{l%
 r<{\,Mb/s}%
 r<{\,Mb/s}%
 r<{\,Mb/s}%
 r<{\,\%}%
}
Agent  				& Eingehende		& Ausgehende		& Gesamt		& Auslastung von Eth0	\\
\hline
DotA				& 0,0769		& 0,1222		& 0,19900		& 0,0199 		\\
Dazzle 				& 8,04 			& 8,62			& 16,66 		& 1,666			\\
Tusk 				& 8,48			& 8,37			& 16,85			& 1,685			\\
Tinker				& 8,42			& 8,56			& 16,98			& 1,698			\\
Lion				& 8,42			& 8,65			& 16,89			& 1,707			\\ 
Agent $\diameter $	 	& 8,34			& 8,55 			& 16,89			& 1,689 		\\   
Agent \& Server $\diameter$   	& 6,68736		& 6,86444		& 13,5518		& 1,35518		\\ 

\end{tabular}
\caption{Auslastung des Ethernet Ports bei 20 MB Paketen auf allen Pis}
\label{tab:standardTraffic}
\end{table}

In der \cref{tab:standardTraffic} sind die Durchnittswerte f"ur den Traffic der jeweiligen Agents eingespeichert. Auch der %
des Servers, da dieser im selben Netzwerk aufgestellt ist wie die anderen Agents. Die Tabelle zeigt uns auch %
das sich die Agents alle in einem "ahnlichen Umfeld was deren Last angeht. Die Standardabweichung $\sigma$ der Agents %
best"atigt diese Annahme diese liegt nach \cref{tab:standardTrafficAbweichung} bei 0,15411 Megabit.     

\begin{table}
\centering
\begin{tabular}{l%
 r<{\,Mb/s}%
 r<{\,Mb/s}%
 r<{\,Mb/s}%
 r<{\,\%}%
}
Agent		& Eingehende            & Ausgehende            & Gesamt                & Last auf Eth0	 \\
\hline
Agents		& 0,174929              & 0,108857              & 0,15411               &  0,01541    	 \\
Agents \& Server& 3,308981		& 3,372526        	& 6,67782             	&  0,66782   	 \\
\end{tabular}
\caption{Standardabweichung des der Last auf dem Ethernet Port bei 20 MB gro"sen Paketen }
\label{tab:standardTrafficAbweichung}
\end{table}

\begin{table}
\centering
\begin{tabular}{l%
 r<{\,KB/s}%
 r<{\,Ops/s}%
}
Agent	  			& Schreiben	 	& Input/Output 	\\	
\hline
Dazzle 				& 1,04			& 154,09	        \\
Tusk 				& 1,11			& 42,4			\\
Tinker				& 1,07 			& 142,5		 	\\
Lion				& 1,04			& 144,31	 	\\
Agent $\diameter $  		& 1,065			& 120,825		\\   
Agent $\sigma $ 		& 0,028722813  		& 45,4928		\\
\end{tabular}
\caption{I/O Zeiten bei Normalbetrieb auf den Pis}
\label{tab:NormalbetriebIoStat}
\end{table}


In der \cref{fig:IoStatDazzleStandard} sieht man das die Festplatte des Raspberry Pis konstant beschrieben wird, lese Operationen finden "uberhaupt nicht statt. %
Im durchschnitt werden 1.04 Kilobytes die Sekunde geschrieben. Mit einem einer Maximallast von 1.86 Kilobytes die Sekunde. %
Was nicht ann"ahrend die Maximale Schreibgeschwindigkeit der verwendeten %
SanDisk SD Karten ist welche bei 30 Megabyte pro Sekunde \autocite{san:sd}  liegen. 

\begin{table}
\centering
\begin{tabular}{l%
 r<{\,\%}%
 r<{\,\%}%
 r<{\,\%}%
 r<{\,\%}%
 r<{\,\%}%
}
Agent  				& Idle			& User Time		& System Time		& I/O wait Time	& Software IRQ	\\
\hline
Dazzle 				& 25,45			& 37,71			& 22,32 		& 1,42		& 16,10	\\
Tusk 				& 23,26			& 37,41			& 23,57			& 0,22		& 15,54	\\
Tinker				& 23,78			& 35,99			& 22,80			& 1,65		& 15,77	\\
Lion				& 22,82			& 35,99			& 23,15			& 1,73		& 15,84	\\ 
Agent $\diameter $  		& 23,83			& 36,14			& 22,96			& 1,73	 	& 15,81	\\   
Agent $\sigma $			&  0,10			&  0,97			&  0,46			& 0,61		& 0,20      \\
\end{tabular}
\caption{CPU Last Verteilung}
\label{tab:standardTraffic}
\end{table}


