\subsection{20 Megabyte Testlauf}
\label{subsec:20MBTest}
Der erste durchgef"uhrte Test basiert auf 20 Megabyte Paketen, dazu wurde das Hintergrundrauschen %
so eingestellt das die gr"o"se der verschickten Pakete 20 Megabyte betr"agt. 

Wie man in \cref{fig:Eth0DazzleStandard} sehen kann ist der durchschnittlich eingehende Durchsatz auf dem Pi Dazzle 8,04 Megabit %
pro Sekunde. Der ausgehende Traffic betr"agt durchschnittlich 8,62 Megabit pro Sekunde. Rechnet man diese Werte in Bytes pro Sekunde um %
betr"agt der ausgehende Datenstrom 1,0775 Megabyte/s und der durchschnittlich eingehende Datenstrom 1,005 Megabyte/s %
Daraus k"onnen wir schlie"sen das der Ethernet Port von Dazzle unter einer durschnittlichen I/O-Last von 2,0825 Megabyte/s stand. % 
Woraus folgt das der Ethernet Port zu 0,20825\% belastet ist.
\begin{table}
\centering
\begin{tabular}{l%
 r<{\,Kb/s}%
 r<{\,Mb/s}%
 r<{\,Mb/s}%
}
Agent  				& Minimal		& Durchschnitt		& Maximal	\\
\hline
DotA				& 7,3			& 0,01218		& 0,0133	\\		
Dazzle 				& 31,64			& 8,13 			& 18,54		\\
Tusk 				& 29,11			& 8,51			& 19,15		\\
Tinker				& 34,3			& 8,33			& 20,28		\\
Lion				& 18,76			& 8,4			& 18,85		\\ 
Agent $\diameter $	 	& 28,4525		& 8,3425		& 19,205	\\   
Agent \& Server $\diameter$   	& 24,22			& 6,68736		& 15,366	\\ 

\end{tabular}
\caption{Eingehender Traffic auf den Ethernet Ports bei 20 MB Paketen auf allen Pis}
\label{tab:EingehenderTraffic20Mb}
\end{table}

\begin{table}
\centering
\begin{tabular}{l%
 r<{\,Kb/s}%
 r<{\,Mb/s}%
 r<{\,Mb/s}%
}
Agent  				& Minimal		& Durchschnitt		& Maximal	\\	
\hline
DotA				& 11,49			& 0,01218		& 0,0133	\\
Dazzle 				& 628,48		& 8,85	 		& 15,02		\\
Tusk 				& 61,18			& 8,47			& 15,35		\\
Tinker				& 421,76		& 8,52			& 13,95		\\
Lion				& 3030			& 8,38			& 13,94		\\ 
Agent $\diameter $	 	& 1035,35		& 8,5555		& 14,565	\\   
Agent \& Server $\diameter$   	& 830,582		& 6,8646436		& 11,65466	\\ 

\end{tabular}
\caption{Ausgehender Traffic auf den Ethernet Ports bei 20 MB Paketen auf allen Pis}
\label{tab:AusgehenderTraffic20Mb}
\end{table}


In der \cref{tab:standardTraffic} sind die Durchnittswerte f"ur den Traffic der jeweiligen Agents eingespeichert. Auch der %
des Servers, da dieser im selben Netzwerk aufgestellt ist wie die anderen Agents. Die Tabelle zeigt uns auch %
das sich die Agents alle in einem "ahnlichen Umfeld was deren Last angeht. Die Standardabweichung $\sigma$ der Agents %
best"atigt diese Annahme diese liegt nach \cref{tab:standardTrafficAbweichung} bei 0,15411 Megabit.     

\begin{table}
\centering
\begin{tabular}{l%
 r<{\,Kb/s}%
 r<{\,Mb/s}%
 r<{\,Mb/s}%
}
Agent				& Minimal		& Durchschnitt          & Max		\\
\hline
Agents $\sigma $		& 5,88919084		& 0,138451255		& 0,6570578361	\\
Agents \& Server $\sigma $	& 9,9666732664		& 3,3344282881        	& 7,6991425381 	\\
\end{tabular}
\caption{Standardabweichung der Eingehende der Last auf dem Ethernet Port bei 20 MB gro"sen Paketen }
\label{tab:standardTrafficAbweichungEingehend20Mb}
\end{table}


\begin{table}
\centering
\begin{tabular}{l%
 r<{\,Kb/s}%
 r<{\,Mb/s}%
 r<{\,Mb/s}%
}
Agent				& Minimal		& Durchschnitt          & Max		\\
\hline
Agents $\sigma $		& 1169,3664626947	& 0,177552809		& 0,6308922253	\\
Agents \& Server $\sigma $	& 1123,2374038359	& 3,420816243        	& 5,8479685073 	\\
\end{tabular}
\caption{Standardabweichung der Ausgehenden der Last auf dem Ethernet Port bei 20 MB gro"sen Paketen }
\label{tab:standardTrafficAbweichungAusgehend20Mb}
\end{table}


\begin{table}
\centering
\begin{tabular}{l%
 r<{\,KB/s}%
 r<{\,Ops/s}%
}
Agent	  			& Schreiben	 	& Input/Output 	\\	
\hline
Dazzle 				& 1,04			& 154,09	        \\
Tusk 				& 1,11			& 42,4			\\
Tinker				& 1,07 			& 142,5		 	\\
Lion				& 1,04			& 144,31	 	\\
Agent $\diameter $  		& 1,065			& 120,825		\\   
Agent $\sigma $ 		& 0,028722813  		& 45,4928		\\
\end{tabular}
\caption{I/O Zeiten bei Normalbetrieb auf den Pis}
\label{tab:NormalbetriebIoStat20Mb}
\end{table}


In der \cref{fig:IoStatDazzleStandard} sieht man das die Festplatte des Raspberry Pis konstant beschrieben wird, lese Operationen finden "uberhaupt nicht statt. %
Im durchschnitt werden 1.04 Kilobytes die Sekunde geschrieben. Mit einem einer Maximallast von 1.86 Kilobytes die Sekunde. %
Was nicht ann"ahrend die Maximale Schreibgeschwindigkeit der verwendeten %
SanDisk SD Karten ist welche bei 30 Megabyte pro Sekunde \autocite{san:sd}  liegen. 

\begin{table}
\centering
\begin{tabular}{l%
 r<{\,\%}%
 r<{\,\%}%
 r<{\,\%}%
 r<{\,\%}%
 r<{\,\%}%
}
Agent  				& Idle			& User Time		& System Time		& I/O wait Time	& Software IRQ	\\
\hline
Dazzle 				& 25,45			& 37,71			& 22,32 		& 1,42		& 16,10	\\
Tusk 				& 23,26			& 37,41			& 23,57			& 0,22		& 15,54	\\
Tinker				& 23,78			& 35,99			& 22,80			& 1,65		& 15,77	\\
Lion				& 22,82			& 35,99			& 23,15			& 1,73		& 15,84	\\ 
Agent $\diameter $  		& 23,83			& 36,14			& 22,96			& 1,73	 	& 15,81	\\   
Agent $\sigma $			&  0,10			&  0,97			&  0,46			& 0,61		& 0,20      \\
\end{tabular}
\caption{CPU Last Verteilung}
\label{tab:CPUlastverteilung20Mb}
\end{table}


