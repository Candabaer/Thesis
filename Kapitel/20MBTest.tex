\subsection{20 Megabyte Testlauf}
\label{subsec:20MBTest}
Der erste durchgef"uhrte Test basiert auf 20 Megabyte Paketen, dazu wurde das Hintergrundrauschen %
so eingestellt das die gr"o"se der verschickten Pakete 20 Megabyte betr"agt. 

Wie man in \cref{fig:Eth0DazzleStandard} sehen kann ist der durchschnittlich ausgehende Traffic auf dem Pi Dazzle 8,85 Mbit/s %
pro Sekunde. Der eingehende Traffic betr"agt durchschnittlich 8,13 Mbit/s. In der \cref{tab:EingehenderTraffic20Mb} %
ist der Datenverkehr der einzelnen Hosts aufgelistet, betrachtet wird der minimale, maximale und der durchschnittlich eingehende Traffic. %
Der Maximal eingehende Datenstrom ist durchschnittlich 19,205 Mbit/s. Der minimal eingehende Traffic, liegt bei %
durchschnittlich 28,45 Kbit/s. Die minimal Werte entstehen, wenn der Pi kein Pakete empf"angt und nur noch mit dem Zabbix Server kommuniziert. %
Dies wird man in die folgenden Tests auch so beobachten k"onnen. Die Standardabweichung der durchschnittlichen %
und der Maximallast sind geringer als 1 Mbit/s. Aus dieser geringen Abweichung vom Durchschnitt kann man schlie"sen das die Hosts %
gleichm"a"sig verteilt Pakete empfangen. % 

\begin{table}
\centering
\begin{tabular}{l%
 r<{\,Kb/s}%
 r<{\,Mb/s}%
 r<{\,Mb/s}%
}
Agent  				& Minimal		& Durchschnitt		& Maximal	\\
\hline
Dazzle 				& 31,64			& 8,13 			& 18,54		\\
Tusk 				& 29,11			& 8,51			& 19,15		\\
Tinker				& 34,3			& 8,33			& 20,28		\\
Lion				& 18,76			& 8,4			& 18,85		\\ 
Agent $\diameter $	 	& 28,4525		& 8,3425		& 19,205	\\   
Agents $\sigma $		& 5,88919084		& 0,138451255		& 0,6570578361	\\

\end{tabular}
\caption{Eingehender Traffic auf den Ethernet Ports bei 20 MB Paketen auf allen Pis}
\label{tab:EingehenderTraffic20Mb}
\begin{tabular}{l%
 r<{\,Kb/s}%
 r<{\,Mb/s}%
 r<{\,Mb/s}%
}
Agent  				& Minimal		& Durchschnitt		& Maximal	\\	
\hline
Dazzle 				& 628,48		& 8,85	 		& 15,02		\\
Tusk 				& 61,18			& 8,47			& 15,35		\\
Tinker				& 421,76		& 8,52			& 13,95		\\
Lion				& 3030			& 8,38			& 13,94		\\ 
Agent $\diameter $	 	& 1035,35		& 8,5555		& 14,565	\\   
Agents $\sigma $		& 1169,3664626947	& 0,177552809		& 0,6308922253	\\

\end{tabular}
\caption{Ausgehender Traffic auf den Ethernet Ports bei 20 MB Paketen auf allen Pis}
\label{tab:AusgehenderTraffic20Mb}

\end{table}

Da, aber auch alle Hosts nicht nur Empf"anger sondern auch gleichzeitig Sender sind wird auch der ausgehende %
Datenstrom betrachtet. Aus der \cref{tab:AusgehenderTraffic20Mb} kann man ablesen das durchschnittlich %
8,56 Mbit/s von jedem einzelnen Host verschickt werden. Wie man sieht betr"agt die Standardabweichung %
$\sigma$ vom ausgehenden Datenstrom 0,18 Mbit/s. Auch die Maximallast hat nur eine geringe Standardabweichung. %
Die Werte der Maximallast und des Durchschnitts "ahneln sehr der vom eingehenden Traffic. %
Jedoch bei der minimallast sieht man das der Host Lion einen Wert von 3,03 Mbit/s aufweist. %
Dadurch erh"oht sich der Durchschnitt der Minimallast drastisch. Man kann auch beobachten %
das Durchschnitt des Maxmimal Ausgehenden Traffics 4 Megabit/s geringer ist als der des eingehenden. %

\begin{table}
\centering
\begin{tabular}{l%
 r<{\,KB/s}%
 r<{\,KB/s}%
 r<{\,Ops/s}%
}
Agent	  			& Schreiben	 	& Lesen			& Input/Output 		\\	
\hline
Dazzle 				& 1,04			& 0			& 154,09	        \\
Tusk 				& 1,11			& 0			& 42,4			\\
Tinker				& 1,07 			& 0			& 142,5		 	\\
Lion				& 1,04			& 0			& 144,31	 	\\
Agent $\diameter $  		& 1,065			& 0			& 120,825		\\   
Agent $\sigma $ 		& 0,028722813  		& 0			& 45,4928		\\
\end{tabular}
\caption{I/O Zeiten bei Normalbetrieb auf den Pis}
\label{tab:NormalbetriebIoStat20Mb}
\end{table}


In der \cref{fig:IoStatDazzleStandard} sieht man das die Festplatte des Raspberry Pis konstant beschrieben wird, lese Operationen finden "uberhaupt nicht statt. %
Im durchschnitt werden 1.07 Kilobytes die Sekunde geschrieben.  %
Was nicht ann"ahrend die Maximale Schreibgeschwindigkeit der verwendeten %
SanDisk SD Karten ist, welche bei 30 Megabyte pro Sekunde \autocite{san:sd} liegt. %
Aus der \cref{tab:NormalbetriebIoStat20Mb} l"asst sich also schlie"sen das die Festplatten der Agents kaum belastet werden. % 

\begin{table}
\centering
\begin{tabular}{l%
 r<{\,\%}%
 r<{\,\%}%
 r<{\,\%}%
 r<{\,\%}%
 r<{\,\%}%
}
Agent  				& Idle			& User Time		& System Time		& I/O wait Time	& Software IRQ	\\
\hline
Dazzle 				& 25,45			& 37,71			& 22,32 		& 1,42		& 16,10	\\
Tusk 				& 23,26			& 37,41			& 23,57			& 0,22		& 15,54	\\
Tinker				& 23,78			& 35,99			& 22,80			& 1,65		& 15,77	\\
Lion				& 22,82			& 35,99			& 23,15			& 1,73		& 15,84	\\ 
Agent $\diameter $  		& 23,83			& 36,14			& 22,96			& 1,26	 	& 15,81	\\   
Agent $\sigma $			&  0,10			&  0,97			&  0,46			& 0,61		& 0,20  \\ %
%
&\end{tabular}
\caption{CPU Last Verteilung}
\label{tab:CPUlastverteilung20Mb}
\end{table}

Betrachtet man nun die \cref{tab:CPUlastverteilung20Mb} spiegelen sich die Ergebnisse aus der \cref{tab:NormalbetriebIoStat20Mb} wieder. %
Die Zeit die die CPU braucht um auf den Abschluss einer Lese oder Schreib Operation zu warten h"alt sich sehr gering. %
So werden durchschnittlich 1,26 \% der CPU Zeit f"ur Lese und Schreib Operationen verwendet. Der Gro"steil der CPU Zeit wird %
bei allen Hosts, daf"ur verwendet die User Anwendungen laufen zu lassen, im Schnitt 36,14 \%. Idle gibt an wie viel der \% vom Prozessor %
ohne eine Aufgabe war. W"ahrend die System Time f"ur die verwaltung von der Netzwerkaufgaben zust"andig war, wie das versenden und das empfangen von %
den Paketen die im Netzwerk verschickt werden. \emph{An Julian: F"ur was ist die Software IRQ time ? Durch was werden die ganzen Software Interrupt %
ausgel"ost ich peils nicht.}%
Wenn man nun die Werte der Standardabweichung der CPU betrachtet best"agtigt sich die Annahme die man aus der % 
\cref{tab:AusgehenderTraffic20Mb} und der \cref{tab:EingehenderTraffic20Mb} gezogen hatte, n"amlich das alle Hosts gleichm"a"sig belastet worden sind. %
Die Standardbweichung der Erwartungswerte geht bei allen Werten nicht "uber 1 \%, welches ein Indikator daf"ur ist das die %
Prozessoren auf den Hosts in etwa die selbe Last tragen.  

Es wurden "uber eine dauer von 12 Stunden 10937 Pakete im Netzwerk verschickt. Die Menge der Daten die verschickt wurden sind 213,61 Gigabyte in diesem Netzwerk . %
Rechnet man dies auf eine Stunde runter kommt man auf 17,8 Gigabyte pro Stunde oder 911,41 Pakete pro Stunde. Wieder l"asst sich eine gute gleichverteilung erkennen. %
Die Standardabweichung der Erfolgreich verschickten Pakete liegt 15 Paketen und auch die Fehlerrate ist sehr gering, ein Host %
hat sogar innerhalb von 12 Stunden kein einziges Paket nicht erfolgreich absenden k"onnen. Generell ist eine sehr geringe Menge an Daten verloren %
gegangen, insgesamt haben 80 Megabyte ihr Ziel nicht Ordnungsgem"a"s erreicht. %  
\begin{table}
\centering
\begin{tabular}{l%
 r<{\,}%
 r<{\,}%
 r<{\,\%}%
 r<{\,GB}%
}
Agent  				& Erfolgreich gesendet			& Erfolglos gesendet			& Erfolglos gesendet	& Verschickte	\\
\hline
Dazzle 				& 2731			 		& 1					& 0,04			& 53,34			\\
Tusk 				& 2710					& 0					& 0,00			& 52,93			\\
Tinker				& 2751					& 2					& 0,07			& 53,73			\\
Lion				& 2745					& 1					& 0,04			& 53,61			\\ 
Summe				& 1093					& 4					& 0,03 			& 213,61		\\
Agent $\diameter $  		& 2734,25				& 1				 	& 0,0375		& 53,40 		\\   
Agent $\sigma $			& 15,76983	 			& 0,70711				& 0,01427      		& 0,308004		\\
\end{tabular}
gesendet\caption{Anzahl der gesendeten Pakete "uber einen Zeitraum von 12 Stunden}
\label{tab:VerschickteDaten20Mb}
\end{table}



