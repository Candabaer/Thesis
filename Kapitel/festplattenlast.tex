\section{Festplatten Auslastung}
\label{sec:festlast}

In diesem Abschnitt wird die Auslastung der Festplatten betrachtet. %
Dabei wird die Metrik der Input Output Operations per Second hinzugezogen. %
Diese gibt an wieviele Lese oder Schreib Operationen in der Sekunde auf der Festplatte %
stattfinden. Je h"oher der Wert ist desto mehr wird auf der Festplatte gelesen oder geschrieben. %
Daraus folgt, jedoch auch das die CPU eine h"ohere I/O Wait Time hat und somit l"anger darauf %
wartet das die Festplatte mit einer Schreib oder Lese Operation abgeschlossen hat. %
Deshalb wird als erstes die CPU I/O Wait Time betrachtet. Wie man aus der \cref{tab:compIoWaitCpu} %
sehen kann ist beim 20 Megabyte Test die geringste Wartezeit mit 1,26 \% f"ur die CPU. Danach gefolgt von dem 200 Megabyte %
Testfall, dieser hat mit 5,03 \% die zweith"ochste I/O Wait Time. Beim 2000 Megabyte Test ist mit 5,85 \% die %
CPU am l"angsten damit besch"aftigt auf den Abschluss einer Lese oder Schreib Operation zu warten. Jedoch ist %
die Differenz zwischen dem 200 und 2000 Megabyte Test nicht so signifikant wie zwischen dem 20 Megabyte Test. %

\begin{table}
\centering
\begin{tabular}{l%
 r<{\,\%}%
 r<{\,\%}%
 r<{\,\%}%
}
Agent	  			& \multicolumn{1}{r}{20 Megabyte}	 	& \multicolumn{1}{r}{200 Megabyte}		& \multicolumn{1}{r}{2000 Megabyte} 		\\	
\hline
Dazzle 				& 1,42						& 7,02						& 7,33	        \\
Tusk 				& 0,22						& 1,20						& 1,27		\\
Tinker				& 1,65						& 5,67						& 8,11	 	\\
Lion				& 1,73						& 6,68						& 6,68	 	\\
Agent $\diameter $  		& 1,26						& 5,14						& 5,85		\\   
Agent $\sigma $ 		& 0,20 						& 2,33						& 2,69			\\
\end{tabular}
\caption{CPU Wartezeit auf den Abschluss einer Lese oder Schreib Operation, Werte aus den \cref{tab:CPUlastverteilung20Mb,tab:CPUlastverteilung200Mb,tab:CPUlastverteilung2000Mb}.}
\label{tab:compIoWaitCpu}
\end{table}

In der \cref{tab:compOps} sind die jeweiligen Lese und Schreib Operationen pro Sekunde der einzelen %
Testf"alle aufgelistet. Wie man sieht geschehen die geringsten I/O Operationen w"ahren dem 20 Megabyte %
Test. Das spiegelt sich auch in der I/O Wait Time der CPU wieder, mit durchschnittlich 120,73 Ops/s %
wird bei diesem Test am wenigstens geschrieben. Wie die \cref{tab:NormalbetriebIoStat20Mb} auch zeigt %
wird in diesem Testfall gar keine Daten gelesen. Das erkl"art auch wieso die Werte so f"ur die Lese %
und Schreib Operationen so gering sind. W"ahrend erst beim 200 Megabyte Test die ersten Lese Operationen %
stattfinden mit durchschnittlich 907,56 Byte/s wie man aus \cref{tab:NormalbetriebIoStat200Mb} lesen kann. Dies erkl"art auch den Anstieg der Lese und Schreib Operationen %
Die der Durchschnitt der I/O Operationen/s zwischen den Tests mit 200 Megabyte Paketen und 2000 Megabyte Paketen %
liegt nahe beinander. Dasselbe ist auch beim Durchschnitt der CPU I/O Wait Time zu beobachten, jedoch %
haben sich die Verh"altnisse ver"andert. W"ahrend beim 200 Megabyte Test mehr geschrieben als gelesen wurde %
ist dies beim Testfall mit 2000 Megabyte andersrum, in 2000 Megabyte Testfall wird mit durchschnittlich 2,30 KB/s
gelesen nach \cref{tab:NormalbetriebIoStat2000Mb}. Daf"ur ist die durchschnittliche Menge die geschrieben wird %
auf 2,12 KB/s, w"ahrend die vom 200 Megabyte Test noch bei 2,65 KB/s lag.   


\begin{table}
\centering
\begin{tabular}{l%
 r<{\,Ops/s}%
 r<{\,Ops/s}%
 r<{\,Ops/s}%
}
Agent	  			& \multicolumn{1}{r}{20 Megabyte}	 	& \multicolumn{1}{r}{200 Megabyte}		& \multicolumn{1}{r}{2000 Megabyte} 		\\	
\hline
Dazzle 				& 153,09					& 263,34					& 257,88	        \\
Tusk 				& 42,40						& 122,64					& 132,75		\\
Tinker				& 142,50					& 234,48					& 257,25	 	\\
Lion				& 144,31					& 248,08					& 248,08	 	\\
Agent $\diameter $  		& 120,73					& 218,39					& 223,99		\\   
Agent $\sigma $ 		& 45,31 					& 56,58						& 52,82			\\
\end{tabular}
\caption{Lese und Schreibzugriffe auf die Festplatten w"ahrend der Testl"aufe, Werte aus den \cref{tab:NormalbetriebIoStat20Mb,tab:NormalbetriebIoStat200Mb,tab:NormalbetriebIoStat2000Mb}.}
\label{tab:compOps}
\end{table}
