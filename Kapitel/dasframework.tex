\chapter{Das Framework}
\label{cha:framework}
\section{Verwendete Hardware} \label{sec:verwendeteHardware}

In diesem Framework wird folgende Hardware verwendet.

\begin{itemize}
\item Zwei Switches
\item Vier Raspberry Pis der ersten Generation
\item Ein Raspberry Pi der zweiten Generation
\item Mehrere Ethernet Kabel
\end{itemize}

Diese Ger"ate wurden in einem eigenst"andigen Netzwerk zusammengeschaltet. So sind an jedem Switch zwei Pis %
der ersten Generation angeschlossen w"ahrend an einem der Switches der Pi der zweiten Generation angeschlossen, %
ist (siehe \cref{fig:AufbauVomNetzwerk}). Welche Software auf den Pis verwendet wurde, wird in \cref{sec:aufbauSoftware} erkl"art. 

\begin{table}
\centering
\begin{tabular}{l%
 r<{\,MHz}%
 r<{\,MB}%
 r<{\,GB}%
 r<{\,MB/s}%
 r<{\,Mbit/s}%
}
Ger"at 		& CPU	& Arbeitspeicher	& Speicher	& Netzwerk Port	\\
\hline
Raspberry Pi 1	& 700	& 512			& 8		& 100		\\
Raspberry Pi 2 	& 700 	& 1024			& 32		& 100		\\

\end{tabular}
\caption{Spezifikation der Raspberry Pis}
\label{tab:hardwarespezifiktion}
\end{table}


Die beiden verwendeten Switches Arbeiten mit 100 oder 1000 Mbit/s, doch aufgrund %
der verwendeten Ethernet Ports auf den Raspberry Pis, welche eine Ubertrangsungrate von 100 Mbit/s haben %
ist der Austausch der Pakete zwischen den Pis, auf 100 Mbit/s beschr"ankt und kann so das ganze Potential der %
Switches nicht ausnutzen. %


 
\section{Aufbau der Software} \label{sec:aufbauSoftware}
Das Testframework besteht aus drei verschiedenen Teilen. %
\begin{itemize}
\item Zabbix %
\item Eigenentwicklung auf dem Server %
\item Eigenentwicklung auf dem Agent %
\end{itemize}
Die Eigenentwicklungen sind alle mit Bash Skript programmiert, Zabbix ist eine bereits fertige Open Source L"osung %
Im folgenden Abschnitt wird ein Einblick in diese beiden Komponenten gegeben. %

\subsection{Zabbix}
Zabbix ist ein Open Source Netzwerk Monitoring System. Die erste Version wurde von \mbox{Alexei} Vladishev entwickelt \autocite{zabbix:Web}. % 
Ein weiterer bekannter Vertreter der Netzwerk Monitor Systeme ist Nagios \autocite{wiki:Nagios}, welches wie Zabbix unter der GPL vertrieben %
wird. Womit jedem frei steht Zabbix zu ver"andern und zu erweitern.  % 
Beide Systeme basieren auf einer Client-Server Architektur. Im weiteren wird jedoch nur Zabbix betrachtet. %
Welche aus zwei Komponenten besteht. %

\begin{description}
\item[Zabbix Server]Der Server hat eine auf PHP basierende Weboberfl"ache, "uber die es f"ur den Benutzer m"oglich ist, die Agents zu %
konfigurieren. So k"onnen manuell die Templates erstellt werden, die den Zabbix Agents mitteilen, welche Informationen dem %
Server zu "ubermitteln sind. Einen genaueren Einblick in den Zabbix Server gibt es in \cref{sec:server}. %
\item[Zabbix Agent]Die Clients, die im Netzwerk "uberwacht werden sollen, sind die sogenannten Agents, die  %
Informationen an den Server weiterleiten, die vom Server gefordert werden. In den sp"ateren Kapiteln wird der Hauptfokus auf der %
Auslastung der Festplatte, CPU und des Ethernet Ports liegen. %
Einen genaueren Einblick in den Zabbix Agent gibt es in \cref{sec:agent}. % 
\end{description} 

\subsection{Eigenentwicklung}
Die selbstentwickelte Software wird in zwei Kategorien unterteilt. Ein Teil der Software l"auft %
auf den Agents. Diese haben den Zweck Netzwerk Traffic zu erzeugen. Dadurch entsteht auf dem Netzwerk und %
auf den Zabbix Agents eine Nutzlast, die vom Zabbix Server gesammelt werden kann. Der zweite Teil der Software %
l"auft auf dem Server, die Software auf dem Server unterst"utzt den Entwicklungsprozess auf den Agents. %
\begin{enumerate}
\item Zabbix Server 
\begin{description}
\item[Update Script:]Dieses Script wird von der Software Entwicklungsmaschine ausgef"uhrt. Es aktualisiert den Code auf den Endger"aten, %
die die Programme Hintergrundrauschen, Pinger, Synchronize und Startrauschen aktualisieren. Dabei wird mittels Secure Copy %
der Quellcode auf das Endger"at gespielt.

\item[Get logs:]Die Skripte Pinger und Hintergrundrauschen erstellen jeweils auf den Endger"aten Logfiles. %
Da es jedoch ein sehr hoher Verwaltungsaufwand w"are, auf den Endger"aten die Logfiles weiterzuverwerten, %
werden die auf den Agents gelagerten Logfiles mit dem Skript Get Logs auf dem Rechner gesammelt, der dieses startet. %
Somit hat man die von den Endger"aten gesammelten Logfiles auf einem Rechner und kann %
mit der Weiterverarbeitung der Logfiles beginnen.      

\end{description}

\item Zabbix Agent

\begin{description}
\item[Hintergrundrauschen:]Diese Eigenentwicklung stellt mit Zabbix die Kernkomponente des Frameworks dar. Wie %
der Rest der selbstentwickelten Software, wurde sie komplett in Bash programmiert, da das Framework ausschlie"slich %
in einer Linux Umgebung entwickelt, getestet und verwendet wird. Hintergrundrauschen schickt "uber Secure Copy, %
Pakete von einem Agent zum anderen. Secure Copy baut auf dem SSH Protokoll \autocite{artcl:thorbo} auf %
und baut auf der Secure Shell auf. So wird eine TCP Verbindung zum angesprochenen Host aufgebaut wird und % 
eine Last auf dem Netzwerk und den Endeger"aten erzeugt, die mit Hilfe des Zabbix Servers  %
gemessen werden kann. Au"serdem speichert Hintergrundrauschen die Dauer, die ein Paket ben"otigt, um erfolgreich %
bei seinem zuf"allig ausgew"ahlten Empf"anger anzukommen. Es werden drei verschieden %
gro"se Pakete verschickt: 20 Megabyte, 200 Megabyte und 2 Gigabyte. Die Ergebnisse der Logfiles %
werden in \cref{cha:versuche} betrachetet.

\item[Startrauschen:]Es wird automatisch auf den Endger"aten ausgef"uhrt. Es dient als eine Zeitschaltuhr %
und erm"oglicht ein zeitversetztes Starten des Scripts Hintergrundrauschen. %
Jedoch wird in den in dieser Ausarbeitung betrachteten Tests immer ein zeitgleicher Start %
durchgef"uhrt. Trotzdem wird dieses Skript weiterhin verwendet, da es eine einfache %
M"oglichkeit darstellt, die Tests zu erweitern. %

\item[Synchronize:]Raspberry Pis besitzen keine eigene Batterie wie es handels"ubliche Rechner haben. %
Deshalb ist nach jedem Neustart die Uhrzeit der Pis unzuverl"assig. Dieses Programm synchronisiert %
die Uhrzeiten der Agents mit der Zeit des Zabbix Servers, welcher zwischen den Tests nicht neu gestartet wird. %
Sychnronize baut eine Verbindung mit dem Pi DotA auf und fragt von diesem die Uhrzeit ab. %
Dies geschieht "uber eine Secure Shell Verbindung zum Zabbix Server % 
\begin{verbatim}
TIMESERVER=192.168.2.116
DATE=`sshpass -p 'raspberry' ssh 192.168.2.116 "date +%s"`
sudo date  -s @$DATE 
\end{verbatim}

\item[Pinger:]Pinger wird zusammen mit dem Hintergrundrauschen ausgef"uhrt und ist um den Linux eigenen Ping Befehl %
herum aufgebaut. Mittels Pinger wird die Latenz unter den einzelnen Endger"aten gemessen. %
\begin{verbatim}
Ping 192.168.2.250 | while read pong; 
do echo "[$(date)] $pong"; 
done >> ~/logfiles/ping/TinkerPing20MB &
\end{verbatim}
Wie man sieht, wird als Erstes der Ping befehl ausgef"uhrt. "Uber die Pipe wird dieser jedoch weitergeleitet und in %
einer Schleife weiterverarbeitet. Neben der Meldung, die vom Befehl Ping Befehl kommt, wird noch ein Datum vorgestellt. %
Diese ganze Meldung wird dann in einer Logfile Datei abgespeichert. %
\end{description}
\end{enumerate}
\section{Einsatz des Frameworks} 

Mit dem Einsatz der im vorherigen Abschnitt vorgestellten Software ist das Testframework aufgebaut. In \cref{fig:AufbauVomNetzwerk} %
wird dargestellt, wie die einzelnen Komponenten zusammenspielen. Hier sieht man, dass an einem Switch zwei Raspberry Pis %
und an einem anderen Switch drei Raspberry Pis angeschlossen sind. Einer von diesen Pis ist der Zabbix Server, der %
die aktiven Hosts im Netzwerk "uberwacht. Um das Framework zu starten, muss man als Erstes den Zabbix Server DotA starten. %
Wenn dieser mit dem Boot Vorgang abgeschlossen hat, kann man die restlichen Raspberry Pis einschalten. %
In den Agents wird nun als Erstes das Programm Synchronize ausgef"uhrt. Da Raspberry Pis keine eigene Uhr haben, aber %
die Logfiles und der Zabbix Agent von der Zeit abh"angig sind um korrekt arbeiten zu k"onnen, m"ussen die Uhren synchronisiert werden. %
\label{sec:einsatzImNetzwerk}%
\begin{figure}[htbp]%
\centering%
\includegraphics*[width=0.9\linewidth]{Abb/Netzschaltung3}%
%
\caption{Aufbau des Netzwerks}%
\label{fig:AufbauVomNetzwerk}%
\end{figure}%
Das Skript Synchronize wird aus der Autostart Konfigurationsdatei von den Raspberry Pis ausgef"uhrt. Deshalb gibt es auch keine %
Probleme die Uhrzeit zu setzen, da dieses Programm als Super User ausgef"uhrt wird. %
Nach dem dieses Programm erfolgreich ausgef"uhrt wurde, startet das Startrauschen Skript. %
Dieses Programm erm"oglicht einen zeitversetzten von Hintergrundrauschen und Pinger. %
In Hintergrundrauschen m"ussen jedoch immer kleine Ver"anderungen vorgenommen werden. %
So muss je nach durchzuf"uhrendem Test eine Zeile umgeschrieben werden. %
\begin{verbatim}
SIZE=500
\end{verbatim}
Diese Variablendeklaration muss immer dem auszuf"uhrenden Test angepasst werden. %
Da die Datei die verschickt wird, "uber den Linux internen Befehl \emph{duplicate data} geschrieben %
wird, muss man diesem einen Blockgr"o"se und eine Anzahl an zu schreibenden Bl"ocken mitgeben. %
\begin{verbatim}
dd if=/dev/urandom of=$MYRANDOMFILE bs=4M count=$FILESIZE 
\end{verbatim}
Die Blockgr"o"se betr"agt immer 4 MB, "uber die Variable SIZE kann man die Anzahl der Bl"ocke %
bestimmen. W"urde man zum Beispiel SIZE=10 setzen, w"urde der \emph{duplicate data} Befehl einen 40 Megabyte gro"sen Block erzeugen. %
Im Fall der in dieser Ausarbeitung betrachteten Testf"alle wurde die SIZE 5, 50 und 500 gew"ahlt, welche %
dann eine 20 Megabyte, 200 Megabyte und 2000 Megabyte gro"se Datei erzeugen. %
Ist dieser Prozess abgeschlossen, beginnt das Framework zu arbeiten. Die Daten %
werden zwischen den Agents verschickt und man kann mit der Auswertung beginnen. % 


