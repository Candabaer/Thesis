\chapter{Das Framework}
\label{cha:framework}
\section{Verwendete Hardware} \label{sec:verwendeteHardware}
Die in \cref{sec:versuchbeschreibung} beschriebenen Versuche wurden mit folgender Hardware gemacht. %
\begin{itemize}
\item Zwei Switches
\item Vier Raspberry Pis der ersten Generation
\item Ein Raspberry Pi der zweiten Generation
\item Mehrere Ethernet Kabel
\end{itemize}
Diese Ger"ate wurden in einem eigenst"andigend Netzwerk zusammengeschaltet. So sind an jedem Switch zwei Pis %
der ersten Generation angeschlossen w"ahrend an einem der Switches der Pi der zweiten Generation angeschlossen %
ist siehe abbildung hier. Welche Software auf den Pis verwendet wurde, wird in \cref{sec:aufbauSoftware} erkl"art. 

\section{Aufbau der Software} \label{sec:aufbauSoftware}
Das Testframework besteht im Grunde aus drei verschieden Teilen. %
\begin{enumerate}
\item Zabbix %
\item Eigentwicklung auf dem Server %
\item Eigentwicklung auf dem Agent %
\end{enumerate}
Die Eigentwicklungen sind alle in Bash programmiert, w"ahrend Zabbix eine bereits fertige Open Source L"osung %
ist. In den folgenden Abschnitten werde ich einen Einblick in diese Teile geben. %

\subsection{Zabbix}
Zabbix ist ein Open Source Netzwerk Monitoring System. Die erste Version wurde von Alexei Vladishev entwickelt, welches %
inzwischen von der Firma Zabbix SIA weiterentwickelt wird. % 
Ein weiterer bekannter Vertreter der Netzwerk Monitor Systeme ist Nagios, welches wie Zabbix unter der GPL Lizenz vertrieben wird. % 
Beide Systeme basieren auf einer Client-Server Architektur. Im weiteren wird jedoch nur Zabbix betrachtet. %
Zabbix besteht aus zwei Komponenten. %
\begin{description}
\item[Zabbix Server]Der Server hat eine auf PHP basierende Weboberfl"ache "uber die es f"ur den Benutzer m"oglich ist die Agents zu %
konfigurieren. So k"onnen manuell die Templates erstellt werden die den Zabbix Agents mitteilen welche Informationen, dem %
Server zu "ubermitteln sind. %
"Uber die API Schnittstelle ist es m"oglich Programme f"ur den Server zu schreiben und diese auszuf"uhren. %
Der Zabbix Server selber ist auch ein Agent welcher sich selber "uberwacht. Dieser ist jedoch nicht beteiligt %
im Allgemeinen Prozess des Frameworks.
\item[Zabbix Agent]Die Clients die im Netzwerk "uberwacht werden sollen sind sogenannte Agents die an den Server %
die Informationen weiterleiten, die vom Server gefordert werden. 
\end{description} 

\subsection{Eigenentwicklung}
Die Selbstentwickelte Software wird in zwei Kategorien unterteilt so l"auft ein Teil der Software %
auf den Agents diese haben den Zweck Netzwerk Traffic zu erzeugen und somit Daten die vom Server %
gesammelt werden kann.
\begin{enumerate}
\item Zabbix Server 
\begin{description}
\item[Update Script]Dieses Script wird von der Software Entwicklungsmaschine ausgef"uhrt. Es aktualisiert den Code auf den Endger"aten %
die die Programme Hintergrundrauschen Pinger und Startrauschen ausf"uhren. Dabei wird mittels Secure Copy %
der Quellcode auf das Endger"at gespielt

\item[Get logs]Die Skripte Pinger und Hintergrundrauschen erstellen, jeweils auf den Endger"aten logfiles. %
Da es jedoch ein sehr hoher Verwaltungsaufwand w"are auf den Endger"aten die Logfiles weiterzuverwerten %
werden mit dem Skript GetLogs die dezentralisiert gelagerten Logfiles auf dem Rechner der dieses Skript %
gestartet gesammelt. Somit hat man die von den Endger"aten gesammelten Daten auf einem Rechner und kann %
dessen weiterverarbeitung betreiben.      

\item[Calculate Average]Berechnet den Durschnitt der dauer die ein Paket brauch um erfolgreich versandt %
zu werden.
 
\end{description}
\item Zabbix Agent

\begin{description}
\item[Hintergrundrauschen]Diese Eigentwicklung stellt mit Zabbix die Kernkomponente des Frameworks dar. Wie %
der Rest der selbstentwickelten Software wurde sie komplett in Bash programmiert da das Framework ausschlie"slich %
in einer Linux Umgebung entwickelt, getestet und verwendet wurde. Hintergrundrauschen schickt "uber die Linux %
Standardbefehle Secure Copy, eine Abwandlung der SSH, Pakete von einem Endeger"at zum anderen. So wird eine %
Last auf dem Netzwerk erzeugt die mithilfe von Zabbix gemessen werden kann. Ausserdem speichert dieses Programm %
die Dauer bis ein Paket erfolgreich bei seinem, zuf"allig ausgew"ahlten Empf"anger angekommen ist und dessen gr"o"se. %

\item[Startrauschen]wird automatisch zum Start eines der Endger"ate ausgef"uhrt. Es dient als eine Zeitschaltuhr %
und erm"oglicht ein Zeitversetztes starten des Scripts Hintergrundrauschen, wodurch man den sequentiellen %
Anstieg an Last im Ethnernet beobachten kann.

\item[Synchronize] Raspberry Pis besitzen keine eigene Batterie wie es handels"ubliche Rechner haben %
deshalb ist nach jedem Neustart die Zeit der Pis unzuverl"assig. Dieses Programm aktualisiert %
die Zeiten der Agents und synchronisiert die Zeit der Raspberry Pis mit der vom Zabbix Server. %

\item[Pinger]wird zusammen mit Hintergrundrauschen ausgef"uhrt und ist um den Linux eigenen Ping Befehl %
aufgebaut. Mittels Pinger wird die Latenz unter den einzelnen Endger"aten geme"sen. %
\end{description}
\end{enumerate}
\section{Einsatz im Netzwerk} 

Mit der im vorherigen Abschnitt vorgestellten Software ist das Testframework aufgebaut. In der \cref{fig:AufbauVomNetzwerk} %
wird dargestellt wie die einzelnen Komponenten zusammenspielen. Hier sieht man das an einem Switch zwei Raspberry Pis %
angeschlossen sind und an einem anderen Switch drei Raspberry Pis sind. Einer von diesen Pis ist der Zabbix Server der %
die Aktiven Hosts im folgenden Agents/Zabbix Agents im Netzwerk "uberwacht. 

\label{sec:einsatzImNetzwerk}
\begin{figure}[htbp]
\centering
\includegraphics*[width=0.9\linewidth]{Abb/Netzschaltung3}

\caption{Aufbau vom Netzwerk}
\label{fig:AufbauVomNetzwerk}
\end{figure}














