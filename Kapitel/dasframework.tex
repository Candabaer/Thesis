\chapter{Das Framework}
\label{cha:framework}
\section{Verwendete Hardware} \label{sec:verwendeteHardware}
Die in \cref{sec:testbeschreibung} beschriebenen Versuche wurden mit folgender Hardware durchgef"uhrt. %
\begin{itemize}
\item Zwei Switches
\item Vier Raspberry Pis der ersten Generation
\item Ein Raspberry Pi der zweiten Generation
\item Mehrere Ethernet Kabel
\end{itemize}
Diese Ger"ate wurden in einem eigenst"andigend Netzwerk zusammengeschaltet. So sind an jedem Switch zwei Pis %
der ersten Generation angeschlossen w"ahrend an einem der Switches der Pi der zweiten Generation angeschlossen %
ist (siehe \cref{fig:AufbauVomNetzwerk}). Welche Software auf den Pis verwendet wurde, wird in \cref{sec:aufbauSoftware} erkl"art. 

\section{Aufbau der Software} \label{sec:aufbauSoftware}
Das Testframework besteht aus drei verschiedenen Teilen. %
\begin{itemize}
\item Zabbix %
\item Eigentwicklung auf dem Server %
\item Eigentwicklung auf dem Agent %
\end{itemize}
Die Eigentwicklungen sind alle in Bash programmiert, w"ahrend Zabbix eine bereits fertige Open Source L"osung %
ist. In den folgenden Abschnitten werde ich einen Einblick in diese Teile geben. %

\subsection{Zabbix}
Zabbix ist ein Open Source Netzwerk Monitoring System. Die erste Version wurde von Alexei Vladishev entwickelt, welches %
seit 2005 von der Firma Zabbix SIA weiterentwickelt wird \autocite{zabbix:Web}. % 
Ein weiterer bekannter Vertreter der Netzwerk Monitor Systeme ist Nagios, welches wie Zabbix unter der GPL Lizenz vertrieben wird \autocite{wiki:Nagios}. % 
Beide Systeme basieren auf einer Client-Server Architektur. Im weiteren wird jedoch nur Zabbix betrachtet. %
Zabbix besteht aus zwei Komponenten. %
\begin{description}
\item[Zabbix Server]Der Server hat eine auf PHP basierende Weboberfl"ache "uber die es f"ur den Benutzer m"oglich ist die Agents zu %
konfigurieren. So k"onnen manuell die Templates erstellt werden, die den Zabbix Agents mitteilen, welche Informationen dem %
Server zu "ubermitteln sind. %
"Uber die API Schnittstelle ist es m"oglich Programme f"ur den Server zu schreiben und diese auszuf"uhren. %
Der Zabbix Server ist ein sich selbst "uberwachender Agent. Dieser ist jedoch nicht beteiligt %
im allgemeinen Prozess des Frameworks.
\item[Zabbix Agent]Die Clients, die im Netzwerk "uberwacht werden sollen sind sogenannten Agents, die an den Server %
die Informationen weiterleiten, die vom Server gefordert werden. % 
\end{description} 

\subsection{Eigenentwicklung}
Die selbstentwickelte Software wird in zwei Kategorien unterteilt; ein Teil der Software l"auft %
auf den Agents, diese haben den Zweck Netzwerk Traffic zu erzeugen. Dadurch entsteht auf dem Netzwerk und %
auf den Zabbix Agents eine Nutzlast, die vom Zabbix Server gesammelt werden kann. Der zweite Teil der Software %
die auf dem Server l"auft, die Software auf dem Server unterst"utzt den Entwicklungsprozess auf den Agents. %
\begin{enumerate}
\item Zabbix Server 
\begin{description}
\item[Update Script:]Dieses Script wird von der Software Entwicklungsmaschine ausgef"uhrt. Es aktualisiert den Code auf den Endger"aten, %
die die Programme Hintergrundrauschen, Pinger, Synchronize und Startrauschen aktualisieren. Dabei wird mittels Secure Copy %
der Quellcode auf das Endger"at gespielt.

\item[Get logs:]Die Skripte Pinger und Hintergrundrauschen erstellen jeweils auf den Endger"aten logfiles. %
Da es jedoch ein sehr hoher Verwaltungsaufwand w"are auf den Endger"aten die Logfiles weiterzuverwerten, %
werden mit dem Skript Get Logs die auf den Agents gelagerten Logfiles auf dem Rechner der dieses Skript %
startet gesammelt. Somit hat man die von den Endger"aten gesammelten Daten auf einem Rechner und kann %
dessen Weiterverarbeitung betreiben.      

\item[Calculate Average:]Berechnet den Durschnitt der Dauer, die ein Paket braucht, um erfolgreich versendet %
zu werden.
 
\end{description}

\newpage
\item Zabbix Agent

\begin{description}
\item[Hintergrundrauschen:]Diese Eigentwicklung stellt mit Zabbix die Kernkomponente des Frameworks dar. Wie %
der Rest der selbstentwickelten Software wurde sie komplett in Bash programmiert, da das Framework ausschlie"slich %
in einer Linux Umgebung entwickelt, getestet und verwendet wird. Hintergrundrauschen schickt "uber Secure Copy, %
Pakete von einem Agent zum anderen. So wird eine %
Last auf dem Netzwerk erzeugt, die Mithilfe des Zabbix Servers gemessen werden kann. Ausserdem speichert Hintergrundrauschen %
die Dauer, bis ein Paket erfolgreich bei seinem zuf"allig ausgew"ahlten Empf"anger angekommen ist. %

\item[Startrauschen:]wird automatisch zum Start eines der Endger"ate ausgef"uhrt. Es dient als eine Zeitschaltuhr %
und erm"oglicht ein zeitversetztes Starten des Scripts Hintergrundrauschen, wodurch man den sequentiellen %
Anstieg an Last im Netzwerk beobachten kann.

\item[Synchronize:]Raspberry Pis besitzen keine eigene Batterie wie es handels"ubliche Rechner haben, %
deshalb ist nach jedem Neustart die Uhrzeit der Pis unzuverl"assig. Dieses Programm synchronisiert %
die Uhrzeiten der Agents mit der Zeit des Servers. %

\item[Pinger:]wird zusammen mit dem Hintergrundrauschen ausgef"uhrt und ist um den Linux eigenen Ping Befehl %
aufgebaut. Mittels Pinger wird die Latenz unter den einzelnen Endger"aten gemessen. %
\end{description}
\end{enumerate}
\section{Einsatz im Netzwerk} 

Mit dem Einsatz der im vorherigen Abschnitt vorgestellten Software ist das Testframework aufgebaut. In der \cref{fig:AufbauVomNetzwerk} %
wird dargestellt, wie die einzelnen Komponenten zusammenspielen. Hier sieht man, dass an einem Switch zwei Raspberry Pis %
und an einem anderen Switch drei Raspberry Pis angeschlossen sind. Einer von diesen Pis ist der Zabbix Server, der %
die Aktiven Hosts im Netzwerk "uberwacht. 

\label{sec:einsatzImNetzwerk}
\begin{figure}[htbp]
\centering
\includegraphics*[width=0.9\linewidth]{Abb/Netzschaltung3}

\caption{Aufbau des Netzwerks}
\label{fig:AufbauVomNetzwerk}
\end{figure}














