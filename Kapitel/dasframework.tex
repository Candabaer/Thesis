\chapter{Das Framework}
\label{cha:framework}
\section{Verwendete Hardware} \label{sec:verwendeteHardware}
Die in \cref{sec:versuchbeschreibung} beschriebenen Versuche wurden mit folgender Hardware gemacht. %
\begin{itemize}
\item Zwei Switches
\item Vier Raspberry Pis der ersten Generation
\item Ein Raspberry Pi der zweiten Generation
\item Mehrere Ethernet Kabel
\end{itemize}
Diese Ger"ate wurden in einem eigenst"andigend Netzwerk zusammengeschaltet. So sind an jedem Switch zwei Pis %
der ersten Generation angeschlossen w"ahrend an einem der Switches der Pi der zweiten Generation angeschlossen %
ist siehe abbildung hier. Welche Software auf den Pis verwendet wurde, wird in \cref{sec:aufbauSoftware} erkl"art. 

\section{Aufbau der Software} \label{sec:aufbauSoftware}

\begin{description}
\item[Zabbix]Neben der Eigententwickelten Software wurde in diesem Framework noch das Open Source Netzwerk Monitoring %
Tool Zabbix benutzt. Zabbix wurde von Alexei Vladishev entwickelt, inzwischen liegt die dritte Version %
vor welche auch in diesem Framework verwendet wurde. Ein bekannter vertreter der Netzwerk Monitor Systeme %
ist Nagios, welche unter der GPL vertrieben. Zabbix ist ein Client Server basierendes Monitoring System %
so werden von den zu "uberwachenden Rechnern, den sogenannten Agents, die Daten an den Zabbix Server geschickt und sind dort "uber %
eine Weboberfl"ache abrufbar. "Uber das Webinterface ist es auch m"oglich einzustellen welche vordefinierten Informationen %
von den Agents an den Server verschickt werden sollen. So kann man auch "uber eine Schnittstelle in der %
in der Zabbix API selbstdefinierte Informationen abrufen und sich in der Weboberfl"ache anzeigen lassen, %
wie man sp"ater bei der I/O Last der Endger"ate sehen kann. All diesen Daten werden in einer MySQL Datenbank %
gespeichert und sind "uber die MySQL Standard Schnittstelle abrufbar.

\item[Hintergrundrauschen]Diese Eigentwicklung stellt mit Zabbix die Kernkomponente des Frameworks dar. Wie %
der Rest der selbstentwickelten Software wurde sie komplett in Bash programmiert da das Framework ausschlie"slich %
in einer Linux Umgebung entwickelt, getestet und verwendet wurde. Hintergrundrauschen schickt "uber die Linux %
Standardbefehle Secure Copy, eine Abwandlung der SSH, Pakete von einem Endeger"at zum anderen. So wird eine %
Last auf dem Netzwerk erzeugt die mithilfe von Zabbix gemessen werden kann. Ausserdem speichert dieses Programm %
die Dauer bis ein Paket erfolgreich bei seinem, zuf"allig ausgew"ahlten Empf"anger angekommen ist und dessen gr"o"se. %

\item[Update Script]     

\end{description}
\section{Einsatz im Netzwerk} 
\label{sec:einsatzImNetzwerk}

