\subsection{200 Megabyte Testlauf}
\label{subsec:200MBTest}
Der erste durchgef"uhrte Test basiert auf 20 Megabyte Paketen, dazu wurde das Hintergrundrauschen %
so eingestellt das die gr"o"se der verschickten Pakete 20 Megabyte betr"agt. 

Wie man in \cref{fig:Eth0DazzleStandard} sehen kann ist der durchschnittlich eingehende Durchsatz auf dem Pi Dazzle 8,04 Megabit %
pro Sekunde. Der ausgehende Traffic betr"agt durchschnittlich 8,62 Megabit pro Sekunde. Rechnet man diese Werte in Bytes pro Sekunde um %
betr"agt der ausgehende Datenstrom 1,0775 Megabyte/s und der durchschnittlich eingehende Datenstrom 1,005 Megabyte/s %
Daraus k"onnen wir schlie"sen das der Ethernet Port von Dazzle unter einer durschnittlichen I/O-Last von 2,0825 Megabyte/s stand. % 
Woraus folgt das der Ethernet Port zu 0,20825\% belastet ist.
\begin{table}
\centering
\begin{tabular}{l%
 r<{\,Kb/s}%
 r<{\,Mb/s}%
 r<{\,Mb/s}%
}
Agent  				& Minimal		& Durchschnitt		& Maximal	\\
\hline
DotA				& 7,08			& 0,00766		& 0,0087	\\		
Dazzle 				& 6,2			& 8,49 			& 24,28		\\
Tusk 				& 5,82			& 8,31			& 25,23		\\
Tinker				& 6,12			& 8,14			& 23,47		\\
Lion				& 7,08			& 8,73			& 26,33		\\ 
Agent $\diameter $	 	& 6,055			& 8,4175		& 24,8275	\\   
Agent \& Server $\diameter$   	& 6,26			& 6,735532		& 19,86374	\\ 

\end{tabular}
\caption{Eingehender Traffic auf den Ethernet Ports bei 20 MB Paketen auf allen Pis}
\label{tab:EingehenderTraffic200MB}
\end{table}

\begin{table}
\centering
\begin{tabular}{l%
 r<{\,Kb/s}%
 r<{\,Mb/s}%
 r<{\,Mb/s}%
}
Agent  				& Minimal		& Durchschnitt		& Maximal	\\	
\hline
DotA				& 11,22			& 0,01205		& 0,01328	\\
Dazzle 				& 7,62			& 8,46	 		& 21,11		\\
Tusk 				& 7,66			& 9,22			& 20,98		\\
Tinker				& 7,34			& 8,67			& 20,73		\\
Lion				& 7,63			& 8,32			& 23,05		\\ 
Agent $\diameter $	 	& 7,5625		& 8,6675		& 21,4675	\\   
Agent \& Server $\diameter$   	& 8,294			& 6,93641		& 17,176656	\\ 

\end{tabular}
\caption{Ausgehender Traffic auf den Ethernet Ports bei 20 MB Paketen auf allen Pis}
\label{tab:AusgehenderTraffic200MB}
\end{table}


In der \cref{tab:standardTraffic} sind die Durchnittswerte f"ur den Traffic der jeweiligen Agents eingespeichert. Auch der %
des Servers, da dieser im selben Netzwerk aufgestellt ist wie die anderen Agents. Die Tabelle zeigt uns auch %
das sich die Agents alle in einem "ahnlichen Umfeld was deren Last angeht. Die Standardabweichung $\sigma$ der Agents %
best"atigt diese Annahme diese liegt nach \cref{tab:standardTrafficAbweichung} bei 0,15411 Megabit.     

\begin{table}
\centering
\begin{tabular}{l%
 r<{\,Kb/s}%
 r<{\,Mb/s}%
 r<{\,Mb/s}%
}
Agent				& Minimal		& Durchschnitt          & Max		\\
\hline
Agents $\sigma $		& 0,1423903087		& 0,2187892822		& 1,0679507245	\\
Agents \& Server $\sigma $	& 0,4293250517		& 3,3696231855        	& 9,9733678542 	\\
\end{tabular}
\caption{Standardabweichung der Eingehende der Last auf dem Ethernet Port bei 20 MB gro"sen Paketen }
\label{tab:standardTrafficAbweichungEingehend200MB}
\end{table}


\begin{table}
\centering
\begin{tabular}{l%
 r<{\,Kb/s}%
 r<{\,Mb/s}%
 r<{\,Mb/s}%
}
Agent				& Minimal		& Durchschnitt          & Max		\\
\hline
Agents $\sigma $		& 0,1293010054	 	& 0,3424452511		& 0,9238066627	\\
Agents \& Server $\sigma $	& 1,4675639679		& 3,475702138       	& 8,6213748283 	\\
\end{tabular}
\caption{Standardabweichung der Ausgehenden der Last auf dem Ethernet Port bei 20 MB gro"sen Paketen }
\label{tab:standardTrafficAbweichungAusgehend200Mb}
\end{table}


\begin{table}
\centering
\begin{tabular}{l%
 r<{\,KB/s}%
 r<{\,Ops/s}%
}
Agent	  			& Schreiben	 	& Input/Output 	\\	
\hline
Dazzle 				& 1,04			& 154,09	        \\
Tusk 				& 1,11			& 42,4			\\
Tinker				& 1,07 			& 142,5		 	\\
Lion				& 1,04			& 144,31	 	\\
Agent $\diameter $  		& 1,065			& 120,825		\\   
Agent $\sigma $ 		& 0,028722813  		& 45,4928		\\
\end{tabular}
\caption{I/O Zeiten bei Normalbetrieb auf den Pis}
\label{tab:NormalbetriebIoStat200MB}
\end{table}


In der \cref{fig:IoStatDazzleStandard} sieht man das die Festplatte des Raspberry Pis konstant beschrieben wird, lese Operationen finden "uberhaupt nicht statt. %
Im durchschnitt werden 1.04 Kilobytes die Sekunde geschrieben. Mit einem einer Maximallast von 1.86 Kilobytes die Sekunde. %
Was nicht ann"ahrend die Maximale Schreibgeschwindigkeit der verwendeten %
SanDisk SD Karten ist welche bei 30 Megabyte pro Sekunde \autocite{san:sd}  liegen. 

\begin{table}
\centering
\begin{tabular}{l%
 r<{\,\%}%
 r<{\,\%}%
 r<{\,\%}%
 r<{\,\%}%
 r<{\,\%}%
}
Agent  				& Idle			& User Time		& System Time		& I/O wait Time	& Software IRQ	\\
\hline
Dazzle 				& 25,45			& 37,71			& 22,32 		& 1,42		& 16,10	\\
Tusk 				& 23,26			& 37,41			& 23,57			& 0,22		& 15,54	\\
Tinker				& 23,78			& 35,99			& 22,80			& 1,65		& 15,77	\\
Lion				& 22,82			& 35,99			& 23,15			& 1,73		& 15,84	\\ 
Agent $\diameter $  		& 23,83			& 36,14			& 22,96			& 1,73	 	& 15,81	\\   
Agent $\sigma $			&  0,10			&  0,97			&  0,46			& 0,61		& 0,20      \\
\end{tabular}
\caption{CPU Last Verteilung}
\label{tab:CPUlastverteilung200MB}
\end{table}


