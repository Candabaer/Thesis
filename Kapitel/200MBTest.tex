\section{200 Megabyte Testlauf}
\label{sec:200MBTest}
Der zweite durchgef"uhrte Test basiert auf 200 Megabyte Paketen, dazu wurde das Hintergrundrauschen %
so eingestellt das die gr"o"se der verschickten Pakete 200 Megabyte betr"agt. 

Wie man in \cref{fig:Eth0DazzleStandard} sehen kann ist der durchschnittlich ausgehende Traffic auf dem Pi Dazzle 8,49 Mbit/s. %
Der eingehende Traffic betr"agt durchschnittlich 8,31 Mbit/s. In der \cref{tab:EingehenderTraffic200Mb} %
ist der Datenverkehr der einzelnen Hosts aufgelistet, betrachtet wird der minimale, maximale und der durchschnittlich eingehende Traffic. %
Der Maximal eingehende Datenstrom ist durchschnittlich 19,205 Mbit/s. Der minimal eingehende Traffic, liegt bei %
durchschnittlich 28,45 Kbit/s. Die Standardabweichung der durchschnittlichen Last ist geringer als 1 Megabit/s. %
W"ahrend die der Maximallast gestiegen ist, diese liegt nun bei 1,07 Megabit/s. Daraus kann man schlie"sen das %
einige Pis, mehr belastet werden als andere. Was aber noch keine drastischen Werte sind darauf hinweisen %
das bei der Verteilung der Netzwerk Last ein Pi, einen zu h"oheren Aufwand hat die eingehenden Daten zu %
verarbeiten. 


\begin{table}
\centering
\begin{tabular}{l%
 r<{\,Kb/s}%
 r<{\,Mb/s}%
 r<{\,Mb/s}%
}
Agent  				& Minimal		& Durchschnitt		& Maximal	\\
\hline
Dazzle 				& 6,08			& 8,49 			& 24,28		\\
Tusk 				& 5,82			& 8,31			& 25,23		\\
Tinker				& 6,12			& 8,14			& 23,47		\\
Lion				& 7,04			& 8,73			& 26,33		\\ 
Agent $\diameter $	 	& 6,05			& 8,41			& 24,74		\\   
Agents $\sigma $		& 0,14			& 0,22			& 1,07		\\

\end{tabular}
\caption{Eingehender Traffic auf den Ethernet Ports bei 200 Megabyte Paketen auf allen Pis.}
\label{tab:EingehenderTraffic200Mb}
\begin{tabular}{l%
 r<{\,Kb/s}%
 r<{\,Mb/s}%
 r<{\,Mb/s}%
}
Agent  				& Minimal		& Durchschnitt		& Maximal	\\	
\hline
Dazzle 				& 7,62			& 8,46	 		& 21,11		\\
Tusk 				& 7,66			& 9,22			& 20,98		\\
Tinker				& 7,34			& 8,67			& 20,73		\\
Lion				& 7,63			& 8,32			& 23,05		\\ 
Agent $\diameter $	 	& 7,56			& 8,67			& 21,46		\\   
Agents $\sigma $		& 0,12			& 0,34			& 0,92		\\

\end{tabular}
\caption{Ausgehender Traffic auf den Ethernet Ports bei 200 Megabyte Paketen auf allen Pis.}
\label{tab:AusgehenderTraffic200Mb}

\end{table}

Aus der \cref{tab:AusgehenderTraffic200Mb} kann man ablesen das durchschnittlich %
8,67 Mbit/s von jedem einzelnen Host verschickt werden. Wie man sieht betr"agt die Standardabweichung %
$\sigma$ vom ausgehenden Datenstrom 0,34 Mbit/s. Auch die Maximallast hat nur eine geringe Standardabweichung. %
Die Werte der Maximallast und des Durchschnitts "ahneln der vom eingehenden Traffic. %
Diesmal ist die jedoch die minimale Last vom Host Lion geringer als in \cref{sec:20MBTest}. %
Wodurch kein drastiche Standardabweichung Eintritt und auch der Durchschnittlich ausgehende %
minimal Traffic sich sehr gering h"alt, diesmal im Kilobit/s Bereich, statt wie in \cref{sec:20MBTest} %
im Megabyte Bereich. %  


\begin{table}
\centering
\begin{tabular}{l%
 r<{\,KB/s}%
 r<{\,B/s}%
 r<{\,Ops/s}%
}
Agent	  			& Schreiben	 	& Lesen			& Input/Output 		\\	
\hline
Dazzle 				& 2,71			& 874,04		& 268,34	        \\
Tusk 				& 2,54			& 1160			& 122,64			\\
Tinker				& 2,57 			& 825,75		& 234,48		 	\\
Lion				& 2,76			& 770,47		& 248,08	 	\\
Agent $\diameter $  		& 2,65			& 907,56		& 218,385		\\   
Agent $\sigma $ 		& 0,09  		& 150,27		& 56,576		\\
\end{tabular}
\caption{I/O Zeiten bei 200 Megabyte auf den Pis}
\label{tab:NormalbetriebIoStat200Mb}
\end{table}


In der \cref{fig:IoStatDazzleStandard} sieht man das die Festplatte des Hosts Dazzle nun nicht %
mehr konstant beschrieben wird stattdessen sieht man das inzwischen "uber l"anger Perioden %
ein aussetzen des Schreibens auftritt. durchschnitt werden 1.07 Kilobytes die Sekunde geschrieben.  %
Was nicht ann"ahrend die Maximale Schreibgeschwindigkeit der verwendeten %
SanDisk SD Karten ist, welche bei 30 Megabyte pro Sekunde \autocite{san:sd} liegt. %
Aus der \cref{tab:NormalbetriebIoStat200Mb} l"asst sich also schlie"sen das die Festplatten der Agents kaum belastet werden. % 

\begin{table}
\centering
\begin{tabular}{l%
 r<{\,\%}%
 r<{\,\%}%
 r<{\,\%}%
 r<{\,\%}%
 r<{\,\%}%
}
Agent  				& Idle			& User Time		& System Time		& I/O wait Time	& Software IRQ	\\
\hline
Dazzle 				& 21,83			& 32,71			& 22,49 		& 7,02		& 15,92	\\
Tusk 				& 25,57			& 34,01			& 23,57			& 1,20		& 15,63	\\
Tinker				& 24,55			& 32,05			& 22,18			& 5,67		& 15,53	\\
Lion				& 21,48			& 32,05			& 22,72			& 6,68		& 15,92	\\ 
Agent $\diameter $  		& 23,35			& 33,66			& 22,74			& 5,14	 	& 15,75	\\   
Agent $\sigma $			&  1,75			&  0,98			&  0,52			& 2,33		& 0,17  \\ 

&\end{tabular}
\caption{CPU Last Verteilung bei 200 Megabyte Paketen im Netzwerk}
\label{tab:CPUlastverteilung200Mb}
\end{table}

Betrachtet man nun die \cref{tab:CPUlastverteilung200Mb} spiegeln sich die Ergebnisse aus der \cref{tab:NormalbetriebIoStat200Mb} wieder. %
Bei diesem Test kam es auch zu Leseoperationen und die Anzahl der Schreiboperationen haben sich verdoppelt. %
Das spiegelt sich auch in der I/O wait Time wieder. Die CPU braucht nun ungef"ahr das f"unffache der, CPU-Zeit um %
um eine Lese, oder Schreib Operation abzuschliessen, als in \cref{sec:20MBTest}. %
So wird durchschnittlich 5,14 \% der CPU Leistung f"ur Lese und Schreib Operationen verwendet. Der Gro"steil der CPU Leistung wird %
bei allen Hosts, daf"ur verwendet die User Anwendungen laufen zu lassen, im Schnitt 33,66 \%, welches %
ungef"ahr 3 \% weniger Leistung f"ur die User Anwendungen ist als in \cref{sec:20MBTest}. Die Zeit die die CPU in einem Idle Zustand verbracht %
hat ist ungef"ahr gleich zu den dem Test in \cref{sec:20MBTest} genauso wie die System Time, beide Werte haben sich im Durchschnitt kaum ver"andert. 
Wirft man jetzt einen Blick auf die Standardabweichungen der Hosts, sieht sticht vorallem die I/O Wait time %
heraus. Der Host Tusk, hat einen sehr geringen Wert in der I/O Wait time. W"ahren die anderen Hosts im Schnitt 5,14 \% der CPU Leistung %
liegt die von Tusk bei 1,20 \%. Wie man auch sehen kann hat Tusk auch den h"ochste Leerlauf auf der CPU, System Time und User Time. %  

"Uber eine dauer von 12 Stunden wurden 1151 Pakete im Netzwerk verschickt. Die Menge der Daten die insgesamt verschickt wurden sind 224,80 Gigabyte in diesem Versuchsaufbau. %
Rechnet man dies auf eine Stunde runter kommt man auf 18,73 Gigabyte pro Stunde oder 95,91 Pakete pro Stunde. Wie man sieht ist die Menge an versendeten Daten %
Daten beinahe gleich geblieben. Es wurde knapp 1 Gigabyte an Daten mehr verschickt, und dabei nur 10 \% der Pakete verschickt wie in \cref{sec:20MBTest}. 
Dazu kommt das "uberhaupt kein Paket verloren gegangen ist und alle Daten erfolgreich ihr Ziel erreicht haben. 

\begin{table}
\centering
\begin{tabular}{l%
 r<{\,}%
 r<{\,}%
 r<{\,\%}%
 r<{\,GB}%
}
Agent  				& Erfolgreich gesendet			& Erfolglos gesendet			& Erfolglos gesendet	& Verschickte	\\
\hline
Dazzle 				& 279			 		& 0					& 0			& 54,50			\\
Tusk 				& 290					& 0					& 0			& 56,64			\\
Tinker				& 297					& 0					& 0			& 58,01			\\
Lion				& 285					& 0					& 0			& 55,66			\\ 
Summe				& 1151					& 0					& 0 			& 224,80		\\
Agent $\diameter $  		& 287,75				& 0				 	& 0			& 56,20 		\\   
Agent $\sigma $			& 6,61		 			& 0					& 0      		& 1,29		\\
\end{tabular}
gesendet\caption{Anzahl der gesendeten 200 Megabyte Pakete "uber einen Zeitraum von 12 Stunden}
\label{tab:VerschickteDaten200Mb}
\end{table}


\begin{table}
\centering
\begin{tabular}{l%
 r<{\,}%
 r<{\,}%
 r<{\,}%
 r<{\,}%
 r<{\,}%
}
Agent  				& Dazzle	& Tusk		& Tinker	& Lion		& Summe		\\
\hline
Dazzle 				& 75		& 77		& 78		& 69		& 299		\\
Tusk 				& 67		& 79		& 75		& 59		& 280		\\
Tinker				& 70		& 73		& 56		& 77		& 276		\\
Lion				& 68		& 75		& 70		& 83		& 296		\\ 
Summe				& 280		& 304		& 279 		& 288		& 1151		\\
\end{tabular}
\caption{Die Anzahl der verschickten und der empfangeni 200 Megabyte Pakete pro Host.}
\label{tab:VerschicktePakete200Mb}
\end{table}



