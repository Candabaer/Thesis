\chapter{Versuche}
\label{cha:versuche}
Aufgrund der Unterschiede zwischen einer Virtuellen und realen Maschine k"onnen nicht auf beiden Systemen  %
die gleichen Versuche gemacht. Deshalb m"ussen von den oben vorgestellten Versuchen ein paar Versuche entfernt %
werden da diese in einem realen Netzwerk nicht auftreten. Bei allen Versuchen werden die Daten die vom Zabbix Server %
zur verf"ugung gestellt werden mit mitteln der Stochastik analysiert, vorallem wird ein Fokus auf die Standardabweichung des Traffics %
im Netzwerk gelegt. Als auch auf die Werte die man von der Festplatte sammeln kann. %
So ist es m"oglich die Werte untereinander zu vergleichen. Mit den Ergebnissen %
aus diesem Kapitel werden in \cref{cha:vergleich} die Ergebnisse aus \cref{sec:raspberryPiVersuche} und \cref{sec:VMVersuche} verglichen. % 

\section{Raspberry Pi Versuche}
\label{sec:raspberryPiVersuche}

\subsection{Kein Traffic}
\label{subsec:nix}
In diesem Versuch wird der Normale Netzaufbau benutzt. Jedoch wird als Besonderheit das Skript Hintergrundrauschen %
nicht ausgef"uhrt. So erh"alt man ein Gef"uhl daf"ur wie sich das Netzwerk verh"alt wenn das Netzwerk nicht in Betrieb %
ist. F"ur diesen Test wird als erstes der Zabbix Server betrachtet. Dies ist vorallem interessant, da auch Zabbix selber %
Traffic erzeugt und um sich weiter mit dem Netzwerk ausseinander zu setzen ist es wichtig zu wissen was im Netzwerk passiert %
wenn auf ihm keine Last liegt. %
 
\begin{figure}[htbp]
\centering
\includegraphics*[width=0.9\linewidth]{Abb/Zabbix/ZabbixNoTraffic/ZabbixNoTrafficEth0}

\caption{Traffic auf Eth0 beim Zabbix Server bei keinem Datenaustausch}
\label{fig:Eth0ServerNoTraffic}
\end{figure}

Die \cref{tab:noTraffic} zeigt die durchschnittlichen "Ubertragungswerte der einzelnen Pis auf, wie man sieht
befindet sich der Durchschnitt im geringen Kilobit pro Sekunde Bereich 

\begin{table}
\centering
\begin{tabular}{l%
 r<{\,Kb/s}%
 r<{\,Kb/s}%
 r<{\,Kb/s}%
 r<{\,\%}%
}
Agent	  			& Eingehende		& Ausgehende		& Gesamt		& Auslastung von Eth0	\\
\hline
DotA				& 7,68			& 12,16			& 19,84			& 0,024800000 		\\
Dazzle 				& 2,19	 		& 2,72			& 4,91			& 0,006137500 		\\
Tusk 				& 2,23			& 2,67			& 4,9			& 0,006125000		\\
Tinker				& 2,19	 		& 2,72	 		& 4,91	 		& 0,006137500		\\
Lion				& 2,19	 		& 2,72	 		& 4,91  		& 0,006137500		\\
$\diameter $ Agent 		& 2,20			& 2,37			& 4,9075		& 0.006134375		\\   
$\diameter $ Agent \& Server 	& 3,30  		& 4,33			& 7,894			& 0.009867500		\\
\end{tabular}
\caption{Traffic bei keinem Datenaustausch auf den Pis}
\label{tab:noTraffic}
\end{table}

\begin{table}
\centering
\begin{tabular}{l%
 r<{\,Kb/s}%
 r<{\,Kb/s}%
 r<{\,Kb/s}%
 r<{\,\%}%
}
Agent		& Eingehende            & Ausgehende            & Gesamt                & Last auf Eth0 \\
\hline
Agents		& 0,034641              & 0,043301              & 0,0086600             & 0,000019625  \\
Agents \& Server& 4,903183 		& 8,995337        	& 13,35604            	& 0,089822765        \\
\end{tabular}
\caption{Standarbweichung bei keinem Datenaustausch}
\label{tab:standardTrafficAbweichung}
\end{table}


\subsection{Normalbetrieb}
\label{subsec:normalbetrieb}
Der erste durchgef"uhrte Test ist der Versuch Normalbetrieb. In diesem Test ist das Ziel zu sehen wie %
sich das Netzwerk verh"alt wenn keine St"orungen stattfinden. Dadurch erhalten wir einen Nennwert mit es m"oglich ist %
Aussagen "uber das Netzwerk zu treffen wenn ein St"orungsfall eintritt. Dieser Test wird "uber einen Zeitraum von f"unf %
Stunden absolviert. Die Dateien die im Netzwerk verschickt werden sind alle 20 Megabytes gro"s, dadurch wird ein %
gleichm"a"siger Datenstrom erzeugt. Als erste Kontroll Instanz werden die Daten die vom Zabbix Server erzeugt werden %
"uberpr"uft, da dieses Tool ein f"ur den Endverbraucher praktisches Frontend besitzt. Wie sich ein Agent verh"alt %
wenn im Netz Traffic St"orungsfrei l"auft sieht man in den folgenden zwei Graphen.

\begin{figure}[htbp]
\centering
\includegraphics*[width=0.9\linewidth]{Abb/ZabbixDazzle/Standard/DazzleStandard}

\caption{Traffic auf Eth0 bei Pi Dazzle}
\label{fig:Eth0DazzleStandard}
\end{figure}

Wie man in \cref{fig:Eth0DazzleStandard} sehen kann ist der durchschnittlich eingehende Durchsatz auf dem Pi Dazzle 8,04 Megabit %
pro Sekunde. Der ausgehende Traffic betr"agt durchschnittlich 8,62 Megabit pro Sekunde. Rechnet man diese Werte in Bytes pro Sekunde um %
betr"agt der ausgehende Datenstrom 1,0775 Megabyte/s und der durchschnittlich eingehende Datenstrom 1,005 Megabyte/s %
Daraus k"onnen wir schlie"sen das der Ethernet Port von Dazzle unter einer durschnittlichen I/O-Last von 2,0825 Megabyte/s stand. % 
Woraus folgt das der Ethernet Port zu 20,825\% belastet war.
\begin{table}
\centering
\begin{tabular}{l%
 r<{\,Mb/s}%
 r<{\,Mb/s}%
 r<{\,Mb/s}%
 r<{\,\%}%
}
Agent  				& Eingehende		& Ausgehende		& Gesamt		& Auslastung von Eth0	\\
\hline
DotA				& 0,0076		& 0,0122		& 0,01980		&  0,1987 		\\
Dazzle 				& 1,0050 		& 1,0775		& 2,08250		& 20,8250		\\
Tusk 				& 1,0600		& 1,0463		& 2,10630		& 21,0625		\\
Tinker				& 1,0525		& 1,0700		& 2,12250		& 21,2250		\\
Lion				& 1,0525		& 1,0813		& 2,13380		& 21,3375		\\ 
$\diameter $ Agent 		& 1,0425		& 0,85746 		& 2.11128		& 21.1125 		\\   
$\diameter $ Agent \& Server 	& 0,83552  		& 1,071825		& 1.69298		& 16.92974		\\

\end{tabular}
\caption{Normalbetrieb Traffic auf allen Pis}
\label{tab:standardTraffic}
\end{table}

In der \cref{tab:standardTraffic} sind die Durchnittswerte f"ur den Traffic der jeweiligen Agents eingespeichert. Auch der %
des Servers, da dieser im selben Netzwerk aufgestellt ist wie die anderen Agents. Die Tabelle zeigt uns auch %
das sich die Agents alle in einem "ahnlichen Umfeld befinden was deren Input sowie Output betrifft. So betr"agt die
Standardabweichung der Pis nach der \cref{tab:standardTrafficAbweichung}     

\begin{table}
\centering
\begin{tabular}{l%
 r<{\,Mb/s}%
 r<{\,Mb/s}%
 r<{\,Mb/s}%
 r<{\,\%}%
}
Agent		& Eingehende            & Ausgehende            & Gesamt                & Last auf Eth0 \\
\hline
Agents		& 0,043875              & 0,027196              & 0,03855               &  0,385          \\
Agents \& Server& 0,926675		& 0,938757        	& 1,93351              & 18,709         \\
\end{tabular}
\caption{Normalbetrieb Standarbweichung der Werte}
\label{tab:standardTrafficAbweichung}
\end{table}


\begin{figure}[htbp]
\centering
\includegraphics*[width=0.9\linewidth]{Abb/ZabbixDazzle/Standard/IoStatDazzleStandard}

\caption{I/O--Last auf der Festplatte von Dazzle}
\label{fig:IoStatDazzleStandard}
\end{figure}

In der \cref{fig:IoStatDazzleStandard} sieht man das die Festplatte des Raspberry Pis konstant beschrieben wird. Im durchschnitt werden 1.04 Kilobytes %
die Sekunde geschrieben. Mit einem einer Maximallast von 1.86 Kilobytes die Sekunde. Was nicht ann"ahrend die Maximale Schreibgeschwindigkeit der verwendeten %
SanDisk SD Karten ist welche bei 30 Megabyte\autocite{san:sd} pro Sekunde liegen. 

\subsubsection{Schlussfolgerung Test: Normalbetrieb}
\label{subsubsec:schlussfolgerung:normalbetrieb}
Nach erfolgreichem durchf"uhren des Tests kann man sehen wie sich die einzelnen Endger"ate im Netzwerk verhalten wenn Pakete von 20 MB gr"o"se dar"uber l"auft. %
Man kann sehen das der Datenstrom konstant bleibt und es keine hohen Abweichungen im eingehenden und ausgehen Traffic gibt. Woraus man schlie"sen kann das %
ein Reibungsloser Ablauf im Netzwerk geleistet ist. 




\subsection{Loop}

\label{subsec:loop}
Wie in \cref{sec:testbeschreibung} beschrieben ist ein Loop ein Fehler der auftritt wenn in einem Switch zwei Ports miteinander verbunden worden sind. %
Dasselbe gilt auch wenn zwei Switches miteinander verbunden werden. Loops sind Fehler die auftreten wenn der Endnutzer nicht mit dem Umgang der Hardware vertraut ist, %
oder die IT--Infrastruktur zu gro"s wird. %
Der erwartete Ausgang des Tests ist nach Abschnitt 3.8 \autocite{book:schreiner} ein totaler Ausfall des Netzwerkes. Die nun gezeigten Ergebnisse best"atigen diese %
Annahme.  

\begin{figure}[htbp]
\centering
\includegraphics*[width=0.9\linewidth]{Abb/ZabbixDazzle/Loop/DazzleLoop}

\caption{Traffic auf Eth0 bei einem Loop}
\label{fig:Eth0DazzleLoop}
\end{figure}

Wie man in der Graphik sehen kann bricht die Verbindung zum Zabbix Server komplett ab. Es findet nichtmal der wie in \cref{subsec:nix} aufgezeigte Server Poll statt, % 
der einen geringen Traffic erzeugt. %
Auch der Graph f"ur die Festplatten Last bricht zum Zeitpunkt des Loops ab. Wieder werden im Graphen auch die Durchschnittswerte des Traffics aufgezeigt. %
Wenn man nun die Werte mit denen aus dem \cref{subsec:normalbetrieb} vergleicht stellt man fest das die Durchschnittswerte eine geringe Abweichung aufzeigen. %
Dies liegt dem zugrunde das Zabbix keine neue Daten von den Agents erhalten kann. Da die Berechnung des durchschnittlichen Traffic erst dann erfolgt wenn %
der Agent eine Naricht "uber seinen Zustand versendet. Da diese jedoch nicht den Server erreichen bleiben die Durchschnnittswerte wie beim Normalbetrieb. %

\begin{figure}[htbp]
\centering
\includegraphics*[width=0.9\linewidth]{Abb/ZabbixDazzle/Loop/IoStatDazzleLoop}

\caption{I/O--Last auf der Festplatte bei einem Loop}
\label{fig:IoStatDazzleLoop}
\end{figure}

Hier kann man sehen das der Loop auch die Informationen die der Agent "uber die Festplatte verschickt unerreichbar sind. Dies hat jedoch aus Erfahrungswerten keinen Einflu"s auf die %
Performanz der Agents. Man kann davon ausgehen das sich die I/O--Last der Agents auf den Wert der in \cref{subsec:nix} gezeigten Werte einpendelt. %


\subsubsection{Schlussfolgerung Test: Loop}
\label{subsubsec:schlussfolgerung:loop}

Wie man in \cref{subsec:loop} sehen kann, zeichnet sich ein Loop dadurch aus der komplette Traffic in einem Netzwerk zusammenbricht. In Zabbix selber zeichnet sich das %
dadurch aus das die Graphen einen direkten Schnitt aufzeigen und auch "uber einen l"angeren Zeitraum sich die Durchschnittswerte die von den Agents verschickt werden nicht %
ver"andern. Auch die Logfiles geben aufschluss "uber die Transportierten Pakete so kann man aus diesen lesen das nur noch die Pakete die an den Lokalen Host geschickt werden %
ihr Ziel erreichen.  


\subsection{Doppelte IP}

In diesem Test wurde dem Agent Tinker dieselbe IP vergeben wie sie Dazzle hat. Ohne weiter in den Versuch zu gehen ...

In diesem Test wird der Raspberry Pi Tinker ignoriert da dieser in dem Test keine Pakete erhalten hat und auch vom %
Zabbix Server nicht auffindbar ist wie die \cref{fig:fehlermeldungenTinker} nahelegt.

\begin{figure}[htbp]
\centering
\includegraphics*[width=0.9\linewidth]{Abb/ZabbixTinker/DoppelteIP/TinkerFaults}

\caption{Fehlermeldung auf dem Zabbix Dashboard bez"uglich Tinker}
\label{fig:fehlermeldungenTinker}
\end{figure}

\begin{table}
\centering
\begin{tabular}{l%
 r<{\,Mb/s}%
 r<{\,Mb/s}%
 r<{\,Mb/s}%
 r<{\,\%}%
}
Agent  				& Eingehende		& Ausgehende		& Gesamt		& Auslastung von Eth0	\\
\hline
DotA				& 0,00760		& 0,0122		& 0,01980		&  0,1987 		\\
Dazzle 				& 1,00500		& 1,0775		& 2,08250		& 20,8250		\\
Tusk 				& 1,06000		& 1,0463		& 2,10630		& 21,0625		\\
Lion				& 1,05250		& 1,0813		& 2,13380		& 21,3375		\\
$\diameter $ Agent 		& 1,03917 		& 1,0684 		& 2,10753		& 15,8562 		\\   
$\diameter $ Agent \& Server 	& 0,78128 		& 0,8043		& 1,58560		& 21,0753		\\

\end{tabular}
\caption{Traffic Durchschnittswerte bei Doppelt belegter IP}
\label{tab:DoppelteIPTraffic}
\end{table}


\section{Virtual Machine Versuche}
\label{sec:VMVersuche}
\subsection{Gemachte Tests}
\subsection{Ergebnisse}


