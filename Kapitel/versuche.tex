\chapter{Versuche}
\label{cha:versuche}
Die beste Art ein Netzwerktestframework zu Testen ist in dem man das Framework benutzt %
dazu habe ich am Anfang der Arbeiten selber mehrere Tests definiert, jeder dieser Tests %
ist eine in einem Netzwerk m"ogliche Fehlerquelle die die Leistung des Netzwerkes %
beeintr"achtigen kann. Um Fehler identifizieren %
zu k"onnen m"ussen jedoch auch die Normalwerte bestimmt werden. Dazu muss erstmal ein %
St"orungsfreies Netzwerk aufgebaut werden und dieses muss beobachtet werden. %
Die in diesem Netz gesammelten Werte stellen die Basis f"ur unsere sp"ateren Ergebnisse dar. %
Der zweite Schritt ist es das Netz mit Fehlern aufzubauen und zu beobachten. Dadurch %
haben wir die Werte eines Fehlerbehafteten Netzwerkes und k"onnen diese nun mit dem %
fehlerfreien Netzwerk vergleichen. Dadurch sollte es uns nun m"oglich sein automatisierte %
Aussagen "uber die Art des Fehlers in einem Netzwerk zu treffen. %

\section{Versuchsbeschreibung}
\label{sec:versuchbeschreibung}
\begin{description}
\item[Ethernetkabel ohne Isolierung] Ethernetkabel werden oft sehr strapaziert %
dies kann dazu fuhren das sich die Isolation l"ost und somit Strahlung einwirkt. %
Dies kann die Leistung im Netzwerk beeintr"achtigen. 
\item[Falsch gedrehtes Twisted Pair Kabel] TP Kabel sind die am meisten in einem Ethnernet %
Netzwerk vorkommenden Kabel, es entsteht eine hohe St"orungsicherheit bei TP Kabeln %
sollten jedoch TP Kabel entdrillt worden sein kann eine hohe Strahlenbelastung auftreten. %
\item[Loop] Ein Loop ist wenn ein Switch mit sich selbst verbunden ist, der Switch sendet %
dann st"andig Datagramme an sich selbst und blockiert so jeden Traffic auf der Leitung. %
\item[Nicht angeschlossenes Kabel] Wie verh"alt sich das Netzwerk wenn ein Kabel von einem %
Endger"at im laufenden Betrieb entkoppelt wird.
\item[Forkbomb] Die Forkbomb ist ein Programm das von sich rekursiv Kopien erstellt %
so das der Computer alle sein Resourcen verbraucht nur noch diesen Code auszufuhren. %
\begin{quote} 
:()\{ :|:\&\};: [https://de.wikipedia.org/wiki/Forkbomb] 
\end{quote}
\item[Festplatte l"auft voll] In einem Netzwerk werden st"andig Daten versand, was also %
passiert wenn die Festplatte von einem Rechner voll"auft und der PC somit Arbeitsunf"ahig wird. %
\item[IP Adresse doppelt belegt] Firmennetze benutzen statische IPs damit Webserver immer unter %
dem selben Namen erreichbar bleiben, das jedoch zu konfigurieren kann zu Fehlern Menschlichen %
Fehlern fuhren, was passiert in einem Netz wenn zwei Rechner sich eine IP teilen. %
\item[Kollisionsdom"anen] Kollisionsdom"anen sind bereiche in einem Netzwerk wo sich mehrere Rechner eine %
eine Leitung teilen, dies ist geschieht in der Physikalischen Ebene eines Netwerks. Die Frage ist kann
man eine Kollisionsdom"ane erkennen. 
\end{description}
\section{Raspberry Pi Versuche}
\label{sec:raspberryPiVersuche}
\subsection{Gemachte Tests}
\subsection{Ergebnisse}

\section{Virtual Machine Versuche}
\label{sec:VMVersuche}
\subsection{Gemachte Tests}
\subsection{Ergebnisse}


