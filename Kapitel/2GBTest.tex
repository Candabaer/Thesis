\section{2 Gigabyte Testlauf}
\label{subsec:2GBTest}
Der erste durchgef"uhrte Test basiert auf 20 Megabyte Paketen, dazu wurde das Hintergrundrauschen %
so eingestellt das die gr"o"se der verschickten Pakete 20 Megabyte betr"agt. 

Wie man in \cref{fig:Eth0DazzleStandard} sehen kann ist der durchschnittlich eingehende Durchsatz auf dem Pi Dazzle 8,04 Megabit %
pro Sekunde. Der ausgehende Traffic betr"agt durchschnittlich 8,62 Megabit pro Sekunde. Rechnet man diese Werte in Bytes pro Sekunde um %
betr"agt der ausgehende Datenstrom 1,0775 Megabyte/s und der durchschnittlich eingehende Datenstrom 1,005 Megabyte/s %
Daraus k"onnen wir schlie"sen das der Ethernet Port von Dazzle unter einer durschnittlichen I/O-Last von 2,0825 Megabyte/s stand. % 
Woraus folgt das der Ethernet Port zu 0,20825\% belastet ist.
\begin{table}
\centering
\begin{tabular}{l%
 r<{\,Kb/s}%
 r<{\,Mb/s}%
 r<{\,Mb/s}%
}
Agent  				& Minimal		& Durchschnitt		& Maximal	\\
\hline
DotA				& 7,46			& 0,00792		& 0,01195	\\		
Dazzle 				& 6,14			& 6,39			& 23,54		\\
Tusk 				& 6,17			& 11,86			& 26,03		\\
Tinker				& 6,13			& 5,02			& 21,24		\\
Lion				& 7,64			& 8,63			& 24,98		\\ 
Agent $\diameter $	 	& 6,145			& 7,975			& 23,9475	\\   
Agent \& Server $\diameter$   	& 6,408			& 6,381584		& 19,708016411	\\ 

\end{tabular}
\caption{Eingehender Traffic auf den Ethernet Ports bei 20 MB Paketen auf allen Pis}
\label{tab:EingehenderTraffic2GB}
\end{table}

\begin{table}
\centering
\begin{tabular}{l%
 r<{\,Kb/s}%
 r<{\,Mb/s}%
 r<{\,Mb/s}%
}
Agent  				& Minimal		& Durchschnitt		& Maximal	\\	
\hline
DotA				& 11,58			& 0,01514		& 0,08772	\\
Dazzle 				& 7,71			& 8,48	 		& 22,33		\\
Tusk 				& 7,66			& 5,41			& 19,84		\\
Tinker				& 7,65			& 9,57			& 25,22		\\
Lion				& 7,63			& 9,2			& 24,97		\\ 
Agent $\diameter $	 	& 1035,35		& 8,165			& 23,09		\\   
Agent \& Server $\diameter$   	& 830,582		& 6,535028		& 18,489544	\\ 

\end{tabular}
\caption{Ausgehender Traffic auf den Ethernet Ports bei 20 MB Paketen auf allen Pis}
\label{tab:AusgehenderTraffic2GB}
\end{table}


In der \cref{tab:standardTraffic} sind die Durchnittswerte f"ur den Traffic der jeweiligen Agents eingespeichert. Auch der %
des Servers, da dieser im selben Netzwerk aufgestellt ist wie die anderen Agents. Die Tabelle zeigt uns auch %
das sich die Agents alle in einem "ahnlichen Umfeld was deren Last angeht. Die Standardabweichung $\sigma$ der Agents %
best"atigt diese Annahme diese liegt nach \cref{tab:standardTrafficAbweichung} bei 0,15411 Megabit.     

\begin{table}
\centering
\begin{tabular}{l%
 r<{\,Kb/s}%
 r<{\,Mb/s}%
 r<{\,Mb/s}%
}
Agent				& Minimal		& Durchschnitt          & Max		\\
\hline
Agents $\sigma $		& 0,015			& 2,5868175429		& 1,7957919562	\\
Agents \& Server $\sigma $	& 0,5261710748		& 3,9381719358        	& 9,7080164611 	\\
\end{tabular}
\caption{Standardabweichung der Eingehende der Last auf dem Ethernet Port bei 20 MB gro"sen Paketen }
\label{tab:standardTrafficAbweichungEingehend2GB}
\end{table}


\begin{table}
\centering
\begin{tabular}{l%
 r<{\,Kb/s}%
 r<{\,Mb/s}%
 r<{\,Mb/s}%
}
Agent				& Minimal		& Durchschnitt          & Max		\\
\hline
Agents $\sigma $		& 0,0294745653		& 1,6381773408		& 2,191540554	\\
Agents \& Server $\sigma $	& 1,5672217456		& 3,5740921761        	& 9,4073939873 	\\
\end{tabular}
\caption{Standardabweichung der Ausgehenden der Last auf dem Ethernet Port bei 20 MB gro"sen Paketen }
\label{tab:standardTrafficAbweichungAusgehend2GB}
\end{table}


\begin{table}
\centering
\begin{tabular}{l%
 r<{\,KB/s}%
 r<{\,Ops/s}%
}
Agent	  			& Schreiben	 	& Input/Output 	\\	
\hline
Dazzle 				& 1,04			& 154,09	        \\
Tusk 				& 1,11			& 42,4			\\
Tinker				& 1,07 			& 142,5		 	\\
Lion				& 1,04			& 144,31	 	\\
Agent $\diameter $  		& 1,065			& 120,825		\\   
Agent $\sigma $ 		& 0,028722813  		& 45,4928		\\
\end{tabular}
\caption{I/O Zeiten bei Normalbetrieb auf den Pis}
\label{tab:NormalbetriebIoStat2GB}
\end{table}


In der \cref{fig:IoStatDazzleStandard} sieht man das die Festplatte des Raspberry Pis konstant beschrieben wird, lese Operationen finden "uberhaupt nicht statt. %
Im durchschnitt werden 1.04 Kilobytes die Sekunde geschrieben. Mit einem einer Maximallast von 1.86 Kilobytes die Sekunde. %
Was nicht ann"ahrend die Maximale Schreibgeschwindigkeit der verwendeten %
SanDisk SD Karten ist welche bei 30 Megabyte pro Sekunde \autocite{san:sd}  liegen. 

\begin{table}
\centering
\begin{tabular}{l%
 r<{\,\%}%
 r<{\,\%}%
 r<{\,\%}%
 r<{\,\%}%
 r<{\,\%}%
}
Agent  				& Idle			& User Time		& System Time		& I/O wait Time	& Software IRQ	\\
\hline
Dazzle 				& 25,45			& 37,71			& 22,32 		& 1,42		& 16,10	\\
Tusk 				& 23,26			& 37,41			& 23,57			& 0,22		& 15,54	\\
Tinker				& 23,78			& 35,99			& 22,80			& 1,65		& 15,77	\\
Lion				& 22,82			& 35,99			& 23,15			& 1,73		& 15,84	\\ 
Agent $\diameter $  		& 23,83			& 36,14			& 22,96			& 1,73	 	& 15,81	\\   
Agent $\sigma $			&  0,10			&  0,97			&  0,46			& 0,61		& 0,20      \\
\end{tabular}
\caption{CPU Last Verteilung}
\label{tab:CPUlastverteilung2GB}
\end{table}



Es wurden "uber eine dauer von 12 Stunden 10937 Pakete im Netzwerk verschickt. Die Menge der Daten die verschickt wurden sind 213,61 Gigabyte in diesem Netzwerk . %
Rechnet man dies auf eine Stunde runter kommt man auf 17,8 Gigabyte pro Stunde oder 911,41 Pakete pro Stunde. Wieder l"asst sich eine gute gleichverteilung erkennen. %
Die Standardabweichung der Erfolgreich verschickten Pakete liegt 15 Paketen und auch die Fehlerrate ist sehr gering, ein Host %
hat sogar innerhalb von 12 Stunden kein einziges Paket nicht erfolgreich absenden k"onnen. Generell ist eine sehr geringe Menge an Daten verloren %
gegangen, insgesamt haben 80 Megabyte ihr Ziel nicht Ordnungsgem"a"s erreicht. %  
\begin{table}
\centering
\begin{tabular}{l%
 r<{\,}%
 r<{\,}%
 r<{\,\%}%
 r<{\,GB}%
}
Agent  				& Erfolgreich gesendet			& Erfolglos gesendet			& Erfolglos gesendet	& Verschickte	\\
\hline
Dazzle 				& 2731			 		& 1					& 0,04			& 53,34			\\
Tusk 				& 2710					& 0					& 0,00			& 52,93			\\
Tinker				& 2751					& 2					& 0,07			& 53,73			\\
Lion				& 2745					& 1					& 0,04			& 53,61			\\ 
Summe				& 1093					& 4					& 0,03 			& 213,61		\\
Agent $\diameter $  		& 2734,25				& 1				 	& 0,0375		& 53,40 		\\   
Agent $\sigma $			& 15,76983	 			& 0,70711				& 0,01427      		& 0,308004		\\
\end{tabular}
gesendet\caption{Anzahl der gesendeten Pakete "uber einen Zeitraum von 12 Stunden}
\label{tab:VerschickteDaten20Mb}
\end{table}




