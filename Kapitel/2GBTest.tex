\section{2000 Megabyte Testlauf}
\label{sec:2000MBTest}
Der dritte durchgef"uhrte Test basiert auf 2000 Megabyte Paketen, dazu wurde das Hintergrundrauschen %
so eingestellt das die Gr"o"se der verschickten Pakete 2000 Megabyte betr"agt. 

Wie man in \cref{fig:Standard2000} sehen kann ist der durchschnittlich eingehende Traffic auf dem Pi Dazzle 6,39 Mbit/s. %
Der eingehende Traffic betr"agt durchschnittlich 7,98 Mbit/s. %
Wie man sehen in den \cref{tab:EingehenderTraffic2000Mb,tab:AusgehenderTraffic2000Mb} sehen kann ist jedoch der Traffic von Tusk sehr Abweichend vom Durchschnitt des restlichen Traffic. %
Der Host Tusk hat den h"ochsten durchschnittlich eingehenden Traffic und den h"ochsten Maximalwert %
Im Verlauf dieses Abschnittes wird noch weiter auf diesen Host eingegangen. Der minimal eingehende Traffic liegt bei %
durchschnittlich 28,45 Kbit/s. Die Standardabweichung der durchschnittlichen Last liegt bei 2,58 Megabit/s. %
Dies ist ein Vielfaches mehr als bei dem 20 Megabyte Testlauf und dem 200 Megabyte Testlauf. %
Die Maximallast hat sich zwischen den beiden vorhergehenden Tests eingependelt. Diese liegt bei 23,16 Megabit/s, im Gegensatz zu %
den vorhergangen Tests mit 19,20 Megabit/s und 24,83 Megabit/s. Der Durchschnitt der Minimalwerte liegt mit 6,15 leicht "uber dem  %
der Minimalwerte aus der \cref{tab:EingehenderTraffic200Mb}.     

\begin{table}
\centering
\begin{tabular}{l%
 r<{\,Kb/s}%
 r<{\,Mb/s}%
 r<{\,Mb/s}%
}
Agent  				& Minimal		& Durchschnitt		& Maximal	\\
\hline
Dazzle 				& 6,14			& 6,39 			& 23,54		\\
Tusk 				& 6,17			& 11,86			& 26,03		\\
Tinker				& 6,13			& 5,02			& 21,24		\\
Lion				& 6,14			& 8,63			& 24,98		\\ 
Agent $\diameter $	 	& 6,15			& 7,98			& 23,16		\\   
Agents $\sigma $		& 0,53			& 2,58			& 1,80		\\

\end{tabular}
\caption{Eingehender Traffic auf den Ethernet Ports bei 2000 Megabyte Paketen auf allen Pis.}
\label{tab:EingehenderTraffic2000Mb}
\begin{tabular}{l%
 r<{\,Kb/s}%
 r<{\,Mb/s}%
 r<{\,Mb/s}%
}
Agent  				& Minimal		& Durchschnitt		& Maximal	\\	
\hline
Dazzle 				& 7,71			& 8,48	 		& 22,33		\\
Tusk 				& 7,66			& 5,41			& 19,84		\\
Tinker				& 7,65			& 9,57			& 25,22		\\
Lion				& 7,63			& 9,20			& 24,97		\\ 
Agent $\diameter $	 	& 7,66			& 8,17			& 23,09		\\   
Agents $\sigma $		& 0,03			& 1,64			& 2,19		\\

\end{tabular}
\caption{Ausgehender Traffic auf den Ethernet Ports bei 2000 Megabyte Paketen auf allen Pis.}
\label{tab:AusgehenderTraffic2000Mb}

\end{table}

Betrachtet man nun den ausgehenden Traffic der Hosts, kann man beobachten, dass der Host Tusk %
diesmal mit Abstand den geringsten ausgehenden Traffic hat. Der maximal ausgehende Traffic mit 19,84 Megabit/s %
sowie der durchschnittlich ausgehende Traffic mit 5,41 Megabit/s, liegen unter dem Durchschnitt von 23,09 % 
Megabit/s und 8,17 Megabit/s. Auch die Standardabweichungen, welche in den vorherigen Versuchen sehr gering %
waren, sind bei diesem Test auf "uber 1 Megabit/s gestiegen. Obwohl die Standardabweichung der Minimalwerte %
in keinem anderen Test so niedrig war, sind bei diesem Testlauf die restlichen Standardabweichung die h"ochsten. Deshalb wird das vernachl"assigt, %
da hier ein Netzwerk betrachtet wird, in dem viele Daten verschickt werden sollen. Ein Netzwerk im Ruhezustand, %
entspricht den Anforderungen. %
Deshalb kann man schlussfolgern, dass die Hosts nicht gleichm"a"sig %
belastet worden sind, wie die \cref{tab:EingehenderTraffic2000Mb,tab:AusgehenderTraffic2000Mb} klar zeigen.  % 


\begin{table}
\centering
\begin{tabular}{l%
 r<{\,KB/s}%
 r<{\,KB/s}%
 r<{\,Ops/s}%
}
Agent	  			& Schreiben	 	& Lesen			& Input/Output 		\\	
\hline
Dazzle 				& 2,21			& 2,64			& 257,88	        \\
Tusk 				& 1,60			& 2,86			& 132,75			\\
Tinker				& 1,89 			& 2,92			& 257,25		 	\\
Lion				& 2,76			& 0,77			& 248,08	 	\\
Agent $\diameter $  		& 2,12			& 2,30			& 223,99		\\   
Agent $\sigma $ 		& 0,43  		& 0,89			& 52,82		\\
\end{tabular}
\caption{I/O Zeiten bei 2000 Megabyte auf den Pis.}
\label{tab:NormalbetriebIoStat2000Mb}
\end{table}


In der \cref{fig:IoStatDazzleStandard2000} sieht man, dass die Festplatte des Hosts Dazzle nun nicht %
mehr konstant beschrieben wird. Stattdessen ist ersichtlich, dass inzwischen "uber l"angere Perioden %
ein Aussetzen des Schreibens auftritt. Auch die Perioden in denen der Host liest sind gestiegen. %
Im Durchschnitt werden 1.07 Kilobyte/s geschrieben. Wieder bildet der Host Tusk das %
Schlusslicht mit 1,60 Kilobyte/s. Auch dieser Wert liegt unter dem Durchschnitt. % 


\begin{table}
\centering
\begin{tabular}{l%
 r<{\,\%}%
 r<{\,\%}%
 r<{\,\%}%
 r<{\,\%}%
 r<{\,\%}%
}
Agent  				& Idle			& User Time		& System Time		& I/O wait Time	& Software IRQ	\\
\hline
Dazzle 				& 29,48			& 29,07			& 21,12 		& 7,33		& 12,88	\\
Tusk 				& 12,05			& 42,43			& 26,74			& 1,27		& 15,63	\\
Tinker				& 29,16			& 28,97			& 21,29			& 8,11		& 12,41	\\
Lion				& 29,16			& 30,68			& 22,08			& 6,68		& 16,55	\\ 
Agent $\diameter $  		& 23,67			& 32,79			& 22,81			& 5,85	 	& 14,37	\\   
Agent $\sigma $			&  7,05			&  5,61			&  2,30			& 2,69		& 1,76  \\ 

\end{tabular}
\caption{CPU Lastverteilung bei 2000 Megabyte Paketen im Netzwerk.}
\label{tab:CPUlastverteilung2000Mb}
\end{table}

Die Tabelle \cref{tab:CPUlastverteilung2000Mb} zeigt, dass der Host Tusk einen sehr geringen Anteil der CPU, dem Leerlaufprozess gewidmet hat. %
Mit 12,05 \% liegt dieser weit unter dem Durchschnitt. Das liegt unter anderem an der User Time, welche bei 42,43 \% liegt %
und der System Time, die bei 26,74 \% liegt. Beide Werte liegen deutlich "uber dem Durchschnitt der anderen Hosts, %
w"ahrend die I/O wait Time deutlich unter dem Durchschnitt der anderen Hosts liegt.  Die Gr"unde hierf"ur gehen eindeutig aus der \cref{tab:VerschickteDaten2000Mb} hervor. %

"Uber eine Dauer von 12 Stunden wurden 79 Pakete im Netzwerk verschickt. Die Menge der Daten, die insgesamt verschickt wurden, sind in diesem Versuchsaufbau 154,30 Gigabyte. %
Rechnet man dies auf eine Stunde herunter, kommt man auf 12,85 Gigabyte pro Stunde oder 6,59 Pakete pro Stunde. Wie man sieht, sind das %
knapp 80 Gigabyte weniger als in den vorhergehenden Tests. Dies liegt vorallem daran, dass an den Host Tusk keine Pakete erfolgreich versendet % 
werden konnten. Jeder Versuch dem Host Tusk ein Paket zuzuschicken schlug fehl. W"ahrend alle anderen Hosts keine Pakete verloren haben. %
Der Grund f"ur das Versagen des Verschickens der Pakete an den Host Tusk ist darauf zur"uckzuf"uhren, dass der Host Tusk, laut Zabbix, %
nichtmehr genug freien Speicher zu Verf"ugung hatte. Da dies jedoch bei der Paket"ubertragung erst festgestellt wurde wenn der Vorgang fast %
abgeschlossen worden ist. Lief der Prozess weiter, bis der Host Tusk keine eingehenden Pakete mehr aktzeptierte und die %
bereits empfangenen Pakete wieder verwarf. Da der Prozess trotzdem eine Zeit lang durchgef"uhrt worden ist und da der Sender erwartet hatte %
das die Datei aktzeptiert wird und den Sendeprozess erst sp"at beendete sind dann durch 80 Gigabyte verloren gegangen. % 

\begin{table}
\centering
\begin{tabular}{l%
 r<{\,}%
 r<{\,}%
 r<{\,\%}%
 r<{\,GB}%
}
Agent  				& Erfolgreich gesendet			& Erfolglos gesendet			& Erfolglos gesendet	& Verschickte	\\
\hline
Dazzle 				& 22  			 		& 3  					& 12			& 42,97			\\
Tusk 				& 14  					& 4  					& 22,22			& 27,34			\\
Tinker				& 22  					& 3  					& 12			& 42,97			\\
Lion				& 21  					& 8  					& 27,59			& 41,02			\\ 
Summe				& 79  					& 18  					& 22,78			& 154,30		\\
Agent $\diameter $  		& 19,75					& 4,50				 	& 18,45			& 38,57 		\\   
Agent $\sigma $			& 3,34		 			& 4,50					& 6,73    		& 6,53		\\
\end{tabular}
\caption{Anzahl der gesendeten 2000 Megabyte Pakete "uber einen Zeitraum von 12 Stunden.}
\label{tab:VerschickteDaten2000Mb}
\end{table}

\begin{table}
\centering
\begin{tabular}{l%
 r<{\,}%
 r<{\,}%
 r<{\,}%
 r<{\,}%
 r<{\,}%
}
Agent  				& Dazzle	& Tusk		& Tinker	& Lion		& Summe		\\
\hline
Dazzle 				& 8		& 0		& 9		& 5		& 22		\\
Tusk 				& 10		& 0		& 5		& 7		& 22		\\
Tinker				& 10		& 0		& 5		& 7		& 22		\\
Lion				& 7		& 0		& 6		& 8		& 21		\\ 
Summe				& 35		& 0		& 25 		& 27		& 87		\\
\end{tabular}
\caption{Die Anzahl der verschickten und der empfangenen Pakete pro Host.}
\label{tab:VerschickteDaten2000Mb}
\end{table}

