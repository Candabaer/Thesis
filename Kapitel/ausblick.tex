\chapter{Fazit}
\label{cha:ausblick}

Mit dieser Bachelorarbeit soll bewiesen werden, dass es m"oglich ist, %
f"ur ein gegebenes Netzwerk eine optimale Paketgr"o"se zu bestimmen. %
So dass im Netzwerk eine maximale Menge an Daten "ubertragen wird und %
die Auslastung der im Netzwerk angeschlossen Hosts nicht blockiert. %
Die gesammelten Daten aus den in \cref{cha:framework} vorgestellten Frameworks, %
den verschiedenen Tests aus \cref{cha:versuche} und der Zusammenfassung dieser Daten in %
\cref{cha:ergebnisse} zeigen, dass es m"oglich ist, f"ur einen definierten Verwendungszweck eines %
Netzwerkes eine optimale Paketgr"o"se zu bestimmen.

Damit ist es also auch m"oglich Netzwerke zu testen und rauszufinden welche %
Paketgr"o"sen am besten in ihnen geeignet sind. Da jedoch in dieser Ausarbeitung die Daten %
per Hand verarbeitet wurden. Anstatt die Tests mit einer vordefinierten Menge an Paketgr"o"sen %
durchzuf"uhren. W"are eine sinnvolle Erweiterung des Frameworks, diesen %
kompletten Prozess zu automatisieren dabei sollte jedoch auch die Auswertung der %
so gesammelten Daten automatisiert werden. %

Dazu muss jedoch vorher klar definiert sein welche Anforderungen an das Netzwerk gestellt werden. %
Soll das Netzwerk mit einer hohen Frequenz Pakete verschicken? Oder sollen die Endger"ate im Netzwerk %
keine hohe Belastung der CPU haben?

Desweiteren ist es mit den in dieser Bachelorarbeit geschaffenen Grundlagen auch m"oglich %
in die Richtung der Netzwerkfehlererkennung zu gehen. In \cref{sec:fehler} konnte man %
"uber die gesammelten Daten R"uckschlusse auf einen Fehler machen, %
der die Arbeit im Netzwerk gest"ort hat und auch teilweise Einfluss auf die anderen Hosts genommen hatte. %
Mit weiteren Tests kann man -- auch mit Hilfe des Frameworks -- eine Fehleranalyse betreiben. %
Sammelt man genug Erfahrungswerte kann man so mit dem Framework Fehler in einem Netzwerk %
automatisch erkennen lassen und die Fehler direkt beheben. %

