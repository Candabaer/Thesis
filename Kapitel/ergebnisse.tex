\chapter{Ergebnisse}
\label{cha:ergebnisse}
In diesem Kapitel werden die Ergebnisse aus dem \cref{cha:versuche} miteinander verglichen. %
Dabei wird als erstes die Anzahl der verschickten Pakete und die Menge der verschickten %
Bytes miteinander verglichen und die Zeiten die, ein Paket Paket zum erreichen des %
Zielsystems Ben"otigt. Es wird ein Blick auf den Ping im Netzwerk und die Auslastung auf den %
Ethernet Ports geworfen. Es wird viel Wert auf einen m"oglichst ausgeglichen %
Traffic an den Ports gelegt. Als zweites wird ein Blick auf die Auslastung des Prozessors gelegt. %
Vor allem wird versucht die Paketegr"o"se zu finden, die die CPU am wenigsten in Anspruch nimmt, %
so das der Endnutzer einen m"oglichst Leistungsf"ahigen Host verwenden kann. Als drittes %
und letztes der im \cref{cha:versuche} betrachteten Metriken kommt die Auslastung der Festplatte. %
Bevor es zur Schlussfolgerung kommt wird jedoch noch ein aufgetretener Fehler betrachtet der den %
Host Tusk in \cref{sec:2000MBTest} betroffen hatte. Zum Schluss dieses Kapitels wird eine Schlussfolgerung %
aus den Ergebnissen gezogen. % 

Das Ziel in diesem Kapitel ist es, die Paketgr"o"se zu ermitteln mit der man den h"ochsten Datenaustausch zwischen den %
Hosts erreichen kann, ohne das die Hosts Handlungsunf"ahig werden und Endnutzer noch auf %
den Hosts simple Arbeiten vollbringen k"onnen. Darunter wird gez"ahlt das der Nutzer in der %
Lage ist den Rechner bedienen zu k"onnen und nicht das es auf dem Rechner m"oglich ist aufw"andige Software %
zu verwenden, wie z.B: Photoshop oder "ahnliches.  

\section{Versandte Daten}
\label{sec:verschickteDaten}
Da dieses Framework darauf ausgelegt ist den h"ochsten Datenverkehr zu bestimmen ist %
Auslastung der Ethernet Ports auf den jeweiligen Hosts der erste Indikator um zu bestimmen %
in welcher Paketgr"o"se der h"ochste Austausch an Daten besteht. Dazu werden die durchschnittlichen %
Werte aus der \cref{tab:compTraffic} miteinander verglichen. Diese Durchschnittswerte beziehen sich %
auf die Tabellen aus \cref{cha:versuche} die den Traffic der Ports aufgelistet hatten. Man sieht %
auf den ersten Blick das sich der Traffic sich bei allen drei Tests nur um Kilobyte unterscheidet. %
Bei einer Paketgr"o"se von 200 Megabyte ist jedoch ersichtlich das diese den h"ochsten eingehenden %
Traffic auf den Hosts erzeugt. Bei einer Paketgr"o"se von 2000 Megabyte liegt dieser knapp unter %
8 Megabit/s und ist damit die Paketgr"o"se die den geringsten eingehenden Traffic erzeugt. Bei %
den 20 Megabyte Tests kommen 8,34 Megabit/s durchschnittlich an, dieser liegt im Mittelfeld %
im Vergleich zum eingehenden Traffic. Betrachtet man nun den ausgehenden Traffic ist wieder %
der Test bei dem 200 Megabyte gro"se Pakete verschickt worden sind an der Spitze mit 8,67 %
Megabit/s. Der 2000 Megabyte Test ist diesmal mit 8,17 Megabit/s "uber einen ausgehenden %
Traffic von 8 Megabit/s gekommen bildet aber auch hier wieder das Schlusslicht der "Ubertragung. %
Bei einer Paketgr"o"se von 20 Megabyte ist betr"agt der ausgehende Traffic 8,55 Megabit/s und %
somit auch wieder zwischen den 200 Megabyte und den 2000 Megabyte Paketen. %

\begin{table}
\centering
\begin{tabular}{l%
 r<{\,Mb/s}%
 r<{\,Mb/s}%
}
Tests  					& Eingehender Traffic	& Ausgehender Traffic	\\
\hline
$\diameter$ \hspace{8pt} 20 Megabyte  	& 8,34			& 8,55 			\\
$\diameter$ \hspace{6pt}200 Megabyte  	& 8,42			& 8,67			\\
$\diameter$ 2000 Megabyte	  	& 7,98			& 8,17			\\
\end{tabular}
\caption{Durchschnittlicher eingehender und ausgehender Traffic der einzelnen Testf"alle.}
\label{tab:compTraffic}

\end{table}

Als zweiten vergleichswert wird die Menge an insgesamt verschickten Daten genommen. %
Da es nicht nur das Ziel ist, die Ethernet Ports der Hosts unter der gr"o"stm"oglichen %
Last zu haben sondern das auch effektiv mehr Daten versendet werden sollen, mit denen dann ein anderes %
Hostsystem weiter arbeiten kann ist auch die Menge der verschickten Daten ein entscheidendes Kriterium. Dazu wird die Menge an %
verschickten Daten im Netzwerk miteinander verglichen. Dazu wird in der \cref{tab:compPackages} %
die Anzahl der erfolgreich verschickten Pakete, die verloren gegangen Pakete, die Menge der verschickten %
Byte und der verloren gegangen Byte betrachtet. In der \cref{tab:compPackages} die Ergebnisse aus der %
\cref{tab:VerschickteDaten20Mb}, \cref{tab:VerschickteDaten200Mb} und \cref{tab:VerschickteDaten2000Mb}. %
Bei diesem vergleich ist es offensichtlich welche der Paketgr"o"sen die meisten Pakete verschicken kann, %
hier sieht man das die gr"o"se der Pakete und Anzahl der verschickten Pakete Hand in Hand einhergehen. %
Auch sind vier verlorene Pakete eine sehr geringe Ausfallquote, mit nur 0,04 \% an verloren Paketen, %
ist sind 20 Megabyte Pakete die zweitzuverl"ssigste Paketgr"o"se, was versenden von Daten angeht. %
Am zuverl"assigsten ist in dem Fall der 200 Megabyte Test, in diesem Test ist kein einziges Paket %
verloren gegangen. Anders als im 2000 Megabyte Testfall, bei diesem sind 18 Pakete verloren gegangen, %
was mit 15,56 \% ein im Verh"altnis mit den anderen beiden Tests ein sehr hoher Satz an verlorenen Paketen %
ist. Insgesamt sind bei diesem Test 36 Gigabyte an Daten verloren gegangen. Das ist das 450-fache von dem %
was im 20 Megabyte Test verloren gegangen ist. Damit ist der 2000 Megabyte Testfall nicht der unzuverl"asigste %
von den drei Testf"allen. Auch dieser Testfall schneidet im Vergleich der Menge an verschickten Daten am %
schlechtesten ab, so ist dieser mit 154,30 Gigabyte der Testfall der die wenigsten Informationen %
zwischen den Hosts austauschen konnte. Der Testfall mit den 20 Megabyte Paketen schneidet auch hier %
wieder zwischen dem 2000 Megabyte Test und den 200 Megabyte Test. Dieser hatte insgesamt 213,60 %
Gigabyte "ubertragen. Die meisten "ubertragenen Daten kamen vom 200 Megabyte Testfall mit 224,80 Gigabyte. %

\begin{table}
\centering
\begin{tabular}{l%
 r<{\,}%
   r<{\,}%
r<{\,GB}%
r<{\,}%
}
Tests				& Erfolgreich gesendet			& Erfolglos gesendet 	& Verschickte 	& Verlorene Byte 	\\
\hline
\hspace{8pt} 20 Megabyte  	& 10937					& 4 			& 213,60	& 80 MB		\\
\hspace{6pt}200 Megabyte  	& 1151					& 0			& 224,80	& --		\\
2000 Megabyte	  		& 87					& 18			& 154,30	& 36 GB	\\
\end{tabular}
\caption{Zusammenfassung der Verschickten Pakete und Daten der einzelnen Testf"alle.}
\label{tab:compPackages}
\end{table}

Wie man aus der \cref{tab:compPackages} schlie"sen kann wird im Testfall mit 20 Megabyte Paketen %
die h"ochste Frequenz an verschickten Paketen erreicht mit. Im Testfall mit 200 Megabyte Paketen %
ist Zeitspanne in der ein Paket verschickt wird am zweith"ochsten, gefolgt vom 2000 Megabyte %
Testfall. In der \cref{tab:compZeiten} kann wird genau das aufgezeigt. Dabei wurden die Zeiten %
genommen die in den Logfiles zwischen dem versenden von zwei Paketen gespeichert wurde. Wie man man aus %
dieser Tabelle ablesen kann, wurden durchschnittlich alle 15,80 Sekunden ein 20 Megabyte %
Paket verschickt. Die Standardabweichung f"ur 20 Megabyte Pakete liegt bei 0,11 Sekunden %
woraus man schlie"sen kann das die Pakete zuverl"assig in einem 15,80 $\pm$ 0,11 Sekunden Intervall versendet %
werden. Beim 200 Megabyte Testfall dauerte dies im Schnitt 151,82 Sekunden. Also schon zwei Minuten und 31,82 Sekunden %
was dem 9,61-fachem der Ubertragungsrate des 20 Megabyte Testfalls entspricht. Auch die Standardabweichung ist h"oher %
diese liegt nun bei 3,88 Sekunden, was dem 35,27-fachen der Standardabweichung des 20 Megabyte Testfalls %
entspricht. Betrachtet man nun den 2000 Megabyte Testfall, bemerkt man das die "Ubertragungsdauer eines Paketes %
beim 10,02-fachen des 200 Megabyte Tests liegt oder verglichen mit dem 20 Megabyte Test das 97.10-fache der %
"Ubertragungszeit ben"otigt. Dies spiegelt, sich auch so ungef"ahr in den verschickten Paketen wieder. %
Wie man in der \cref{tab:compPackages} sehen kann entspricht ungef"ahr das 10-Fache der verschickten Pakete im %
2000 Megabyte Test der Anzahl an verschickten Pakete im 200 Megabyte Test. Um das jedoch so zu erreichen m"ussen %
auch die gescheiterten Paket"ubertragungen von dem 2000 Megabyte Test hinzugenommen werden [Wieso dies der Fall ist wird im Abschnitt %
\cref{sec:fehler} erl"autert, Amnk. d. Verf.]. Dasselbe verh"altnis herrscht auch zwischen dem 200 Megabyte Test und %
dem 20 Megabyte Test.  

\begin{table}
\centering
\begin{tabular}{l%
 r<{\,s}%
 r<{\,s}%
 r<{\,s}%
 r<{\,s}%
}
Agent  				& $\diameter $20 Megabyte		& $\diameter $200 Megabyte		& $\diameter $2000 Megabyte		\\
\hline
Dazzle 				& 15,84			 		& 157,65				& 1649,74					\\
Tusk 				& 15,95					& 147,40				& 1445,23				\\
Tinker				& 15,65					& 155,40				& 1574,00				\\
Lion				& 15,75					& 151,82				& 1467,96				\\ 
Agent $\diameter $  		& 15,80					& 153,06			 	& 1534,23				\\   
Agent $\sigma $			& 0,11		 			& 3,88					& 82,52      				\\
\end{tabular}
\caption{Zeiterintervalle bis ein Paket erfolgreich sein Ziel erreicht hat.}
\label{tab:compZeiten}
\end{table}

Als letztes wird der Ping der einzelnen Ger"ate in den verschiedenen Testf"alle betrachtet. %
Obwohl der ICMP Ping auf dem das Programm Pinger aufbaut nicht zuverl"assig ob ein Service %
auf einem Host richtig funktioniert, jedoch gibt es Aufschluss dar"uber ob die Antwort vom eingehende %
ICMP ECHO\_REQUEST stark verz"ogert ist. Sollte die RTT ungew"ohnlich hoch sein, kann es daran liegen %
das der gepingte Host eine sehr hohe CPU-Auslastung hat und den ICMP ECHO\_REQUEST nicht beantworten kann. %
Wie man jedoch aus den \cref{tab:ping20,tab:ping200,tab:ping2000} entnehmen kann ist die Round Trip Time %
sehr gering. Beim 20 Megabyte Testfall Durchschnitt betr"agt die RTT 0,73 ms, w"ahrend beim 200 Megabyte %
Testfall die RTT 0,75 ms betr"agt, womit auch dieser Testfall die hochste RTT hat. Beim 2000 Megabyte Testfall %
liegt die RTT bei 0,73 ms und ist genauso niedrig wie beim 20 Megabyte Testfall. Auff"allig ist jedoch %
das beim Host Lion die h"ochste RTT zu vermessen ist. Betrachtet man die \cref{tab:CPUlastverteilung20Mb,tab:CPUlastverteilung200Mb,tab:CPUlastverteilung2000Mb} %
sieht man das Lion ausser, im 2000 Megabyte Test immer den geringsten Wert im CPU Leerlauf hat. %
Was die hohe RTT von Lion erkl"art. 
  
\begin{table}
\centering
\begin{tabular}{l%
 r<{\,ms}%
 r<{\,ms}%
 r<{\,ms}%
 r<{\,ms}%
 r<{\,ms}%
}
Agent  				& \multicolumn{1}{r}{Dazzle}	& \multicolumn{1}{r}{Tusk}	& \multicolumn{1}{r}{Tinker}	& \multicolumn{1}{r}{Lion}	& \multicolumn{1}{r}{Durchschnitt}		\\
\hline
Dazzle 				& 0,23				& 0,81				& 0,82				& 0,99				& 0,71		\\
Tusk 				& 0,82				& 0,23				& 0,82				& 0,96				& 0,70		\\
Tinker				& 0,83				& 0,81				& 0,23				& 0,99				& 0,71		\\
Lion				& 0,97				& 0,96				& 0,97				& 0,23				& 0,78		\\ 
Durchschnitt			& 0,71				& 0,70				& 0,71 				& 0,79				& 0,73		\\
\end{tabular}
\caption{Die RTT der Pis zueiander beim 20 Megabyte Testfall.}
\label{tab:ping20}
\end{table}

\begin{table}
\centering
\begin{tabular}{l%
 r<{\,ms}%
 r<{\,ms}%
 r<{\,ms}%
 r<{\,ms}%
 r<{\,ms}%
}
Agent  				& \multicolumn{1}{r}{Dazzle}	& \multicolumn{1}{r}{Tusk}	& \multicolumn{1}{r}{Tinker}	& \multicolumn{1}{r}{Lion}	& \multicolumn{1}{r}{Durchschnitt}		\\
\hline
Dazzle 				& 0,23				& 0,85				& 0,84				& 1,00				& 0,73		\\
Tusk 				& 0,85				& 0,24				& 0,85				& 1,02				& 0,74		\\
Tinker				& 0,85				& 0,83				& 0,23				& 1,00				& 0,73		\\
Lion				& 0,99				& 0,98				& 0,99				& 0,23				& 0,80		\\ 
Durchschnitt			& 0,73				& 0,73				& 0,73 				& 0,81				& 0,75		\\
\end{tabular}
\caption{Die RTT der Pis zueiander beim 200 Megabyte Testfall.}
\label{tab:ping200}
\end{table}

\begin{table}
\centering
\begin{tabular}{l%
 r<{\,ms}%
 r<{\,ms}%
 r<{\,ms}%
 r<{\,ms}%
 r<{\,ms}%
}
Agent  				& \multicolumn{1}{r}{Dazzle}	& \multicolumn{1}{r}{Tusk}	& \multicolumn{1}{r}{Tinker}	& \multicolumn{1}{r}{Lion}	& \multicolumn{1}{r}{Durchschnitt}		\\
\hline
Dazzle 				& 0,23				& 0,81				& 0,82				& 0,99				& 0,71		\\
Tusk 				& 0,82				& 0,23				& 0,82				& 0,99				& 0,77		\\
Tinker				& 0,83				& 0,80				& 0,23				& 0,99				& 0,74		\\
Lion				& 0,97				& 0,96				& 0,97				& 0,23				& 0,78		\\ 
Durchschnitt			& 0,71				& 0,70				& 0,71 				& 0,80				& 0,73		\\
\end{tabular}
\caption{Die RTT der Pis zueiander beim 2000 Megabyte Testfall.}
\label{tab:ping2000}
\end{table}

\section{Prozessor Auslastung}
\label{sec:cpulast}
Um die Annahme, die am Ende von \cref{sec:verschickteDaten} gemacht wurde, zu "uberpr"ufen, dass %
ein hohe RTT einer hohen Auslastung der CPU des Zielsystems entpricht, wird nun der Leerlauf den %
die CPUs haben, betrachtet. Da man "uber den Leerlauf R"uckschl"usse "uber die Auslastung der CPU machen kann %
die Theorie die mit der RTT "uber den Host Lion aufgestellt wurde, dass dieser aufgrund der hohen RTT %
unter der h"ochsten Last stand. Wird mit den Werten aus der Tabelle \cref{tab:compCPU} best"agigt. %
Vergleicht man den Leerlauf von Lion mit denen der anderen Hosts, ist der Host Lion derjenige mit %
dem geringsten Leerlauf. Im 2000 Megabyte Testfall jedoch ist die CPU vom Host Tusk am meisten %
ausgelastet. Der Leerlauf beim Host Tusk liegt mit 12,05 \% unter der H"alfte der anderen Hosts im Test. %
Dies ist auch der einzige Test in dem der Host Lion nicht die h"ochste Auslastung hat. Die hohe Auslastung %
des Hosts Tusk h"angt aber auch mit dem Fehler zusammen, der w"ahrend des Tests aufgetreten ist und in dem %
\cref{sec:2000MBTest} und \cref{sec:fehler} schon aufgef"uhrt ist. %

Wie man auch direkt sehen kann sind die drei verbliebenen von den vier Hosts im 2000 Megabyte Test mit einem gro"sen Abstand %
unausgelastet. Mit jeweils 29 \% haben diese Hosts die geringste CPU-Last im Vergleich zu dem 20 Megabyte und %
200 Megabyte Test. Jedoch ist Tusk mit 12,05 \% der ausgelasteste Host in diesem Test. Dies wirkt sich auch auf die %
durchschnittliche Last im Test wieder. Mit 23,67 \% liegt diesmal der 2000 Megabyte Test im Mittelfeld zwischen %
dem 20 Megabyte und 200 Megabyte Test. Bei 20 Megabyte Paketen sind Hosts im Test mit 23,83 \% die mit dem %
h"ochsten Leerlaufprozess. W"ahrend beim 200 Megabyte Test mit \mbox{23,35 \%} der geringste Leerlaufprozess erreicht %
ist. Aus der Standardabweichung der einzelnen Tests kann man auch schlussfolgern, welcher der Tests eine m"oglichst %
gleichm"a"sige Belastung der Hosts hat. So werden beim 20 Megabyte Test, alle Hosts mit einer Standardabweichung %
von $\pm$ 0,10 \% am gleichm"a"sigsten belastet. Gefolgt vom 200 Megabyte Test. Bei diesem Test ist auch noch eine %
geringe Standardabweichung gegeben mit $\pm$ 0,98 \% sind. Auch in diesem Test sind die Hosts gleichm"a"sig belastet. %
Anders als beim 2000 Megabyte Test ist die CPU Lastenverteilung ist sehr ungleich. Mit $\pm$ 7,05 \% ist die Lastverteilung %
am ungleichm"a"sigsten. Der Host Tusk hat in diesem Test die h"ochste Last von allen Host. % 


\begin{table}
\centering
\begin{tabular}{l%
 r<{\,\%}%
 r<{\,\%}%
 r<{\,\%}%
}
Agent  				& \multicolumn{1}{r}{20 Megabyte Idle}			& \multicolumn{1}{r}{200 Megabyte Idle}			& \multicolumn{1}{r}{2000 Megabyte Idle}		\\
\hline
Dazzle 				& 25,45			 				& 21,83							& 29,48					\\
Tusk 				& 23,26							& 25,57							& 12,05					\\
Tinker				& 23,78							& 24,55							& 29,16					\\
Lion				& 22,82							& 21,48							& 29,16					\\ 
Agent $\diameter $  		& 23,83							& 23,35					 		& 23,67					\\   
Agent $\sigma $			& 0,10		 					& 0,98							& 7,05      				\\
\end{tabular}
\caption{Leerlauf der CPUs im vergleich. Werte aus den \cref{tab:CPUlastverteilung20Mb,tab:CPUlastverteilung200Mb,tab:CPUlastverteilung2000Mb}.}
\label{tab:compCPU}
\end{table}


\section{Festplatten Auslastung}
\label{sec:festlast}

In diesem Abschnitt wird die Auslastung der Festplatten betrachtet. %
Dabei wird die Metrik der Input Output Operations per Second hinzugezogen. %
Diese gibt an wieviele Lese oder Schreib Operationen in der Sekunde auf der Festplatte %
stattfinden. Je h"oher der Wert ist desto mehr wird auf der Festplatte gelesen oder geschrieben. %
Daraus folgt, jedoch auch das die CPU eine h"ohere I/O Wait Time hat und somit l"anger darauf %
wartet das die Festplatte mit einer Schreib oder Lese Operation abgeschlossen hat. %
Deshalb wird als erstes die CPU I/O Wait Time betrachtet. Wie man aus der \cref{tab:compIoWaitCpu} %
sehen kann ist beim 20 Megabyte Test die geringste Wartezeit mit 1,26 \% f"ur die CPU. Danach gefolgt von dem 200 Megabyte %
Testfall, dieser hat mit 5,03 \% die zweith"ochste I/O Wait Time. Beim 2000 Megabyte Test ist mit 5,85 \% die %
CPU am l"angsten damit besch"aftigt auf den Abschluss einer Lese oder Schreib Operation zu warten. Jedoch ist %
die Differenz zwischen dem 200 und 2000 Megabyte Test nicht so signifikant wie zwischen dem 20 Megabyte Test. %

\begin{table}
\centering
\begin{tabular}{l%
 r<{\,\%}%
 r<{\,\%}%
 r<{\,\%}%
}
Agent	  			& \multicolumn{1}{r}{20 Megabyte}	 	& \multicolumn{1}{r}{200 Megabyte}		& \multicolumn{1}{r}{2000 Megabyte} 		\\	
\hline
Dazzle 				& 1,42						& 7,02						& 7,33	        \\
Tusk 				& 0,22						& 1,20						& 1,27		\\
Tinker				& 1,65						& 5,67						& 8,11	 	\\
Lion				& 1,73						& 6,68						& 6,68	 	\\
Agent $\diameter $  		& 1,26						& 5,14						& 5,85		\\   
Agent $\sigma $ 		& 0,20 						& 2,33						& 2,69			\\
\end{tabular}
\caption{CPU Wartezeit auf den Abschluss einer Lese oder Schreib Operation, Werte aus den \cref{tab:CPUlastverteilung20Mb,tab:CPUlastverteilung200Mb,tab:CPUlastverteilung2000Mb}.}
\label{tab:compIoWaitCpu}
\end{table}

In der \cref{tab:compOps} sind die jeweiligen Lese und Schreib Operationen pro Sekunde der einzelen %
Testf"alle aufgelistet. Wie man sieht geschehen die geringsten I/O Operationen w"ahren dem 20 Megabyte %
Test. Das spiegelt sich auch in der I/O Wait Time der CPU wieder, mit durchschnittlich 120,73 Ops/s %
wird bei diesem Test am wenigstens geschrieben. Wie die \cref{tab:NormalbetriebIoStat20Mb} auch zeigt %
wird in diesem Testfall gar keine Daten gelesen. Das erkl"art auch wieso die Werte so f"ur die Lese %
und Schreib Operationen so gering sind. W"ahrend erst beim 200 Megabyte Test die ersten Lese Operationen %
stattfinden mit durchschnittlich 907,56 Byte/s wie man aus \cref{tab:NormalbetriebIoStat200Mb} lesen kann. Dies erkl"art auch den Anstieg der Lese und Schreib Operationen %
Die der Durchschnitt der I/O Operationen/s zwischen den Tests mit 200 Megabyte Paketen und 2000 Megabyte Paketen %
liegt nahe beinander. Dasselbe ist auch beim Durchschnitt der CPU I/O Wait Time zu beobachten, jedoch %
haben sich die Verh"altnisse ver"andert. W"ahrend beim 200 Megabyte Test mehr geschrieben als gelesen wurde %
ist dies beim Testfall mit 2000 Megabyte andersrum, in 2000 Megabyte Testfall wird mit durchschnittlich 2,30 KB/s
gelesen nach \cref{tab:NormalbetriebIoStat2000Mb}. Daf"ur ist die durchschnittliche Menge die geschrieben wird %
auf 2,12 KB/s, w"ahrend die vom 200 Megabyte Test noch bei 2,65 KB/s lag.   


\begin{table}
\centering
\begin{tabular}{l%
 r<{\,Ops/s}%
 r<{\,Ops/s}%
 r<{\,Ops/s}%
}
Agent	  			& \multicolumn{1}{r}{20 Megabyte}	 	& \multicolumn{1}{r}{200 Megabyte}		& \multicolumn{1}{r}{2000 Megabyte} 		\\	
\hline
Dazzle 				& 153,09					& 263,34					& 257,88	        \\
Tusk 				& 42,40						& 122,64					& 132,75		\\
Tinker				& 142,50					& 234,48					& 257,25	 	\\
Lion				& 144,31					& 248,08					& 248,08	 	\\
Agent $\diameter $  		& 120,73					& 218,39					& 223,99		\\   
Agent $\sigma $ 		& 45,31 					& 56,58						& 52,82			\\
\end{tabular}
\caption{Lese und Schreibzugriffe auf die Festplatten w"ahrend der Testl"aufe, Werte aus den \cref{tab:NormalbetriebIoStat20Mb,tab:NormalbetriebIoStat200Mb,tab:NormalbetriebIoStat2000Mb}.}
\label{tab:compOps}
\end{table}

\section{Aufgetretene Fehler}
\label{sec:fehler}

Wie schon in \cref{sec:2000MBTest} erw"ahnt kam es bei diesem Test zu einem Fehler %
bei einem der Hostsysteme. Alle Pakete, die zum Host verschickt werden sollten, %
haben ihr Ziel nicht erreicht. Dadurch sind sehr viele Pakete verloren gegangen. %
Insgesamt konnte vom Host Tusk kein einziges Paket angenommen werden. So sind alle %
fehlerhaften Pakete auf ihn zur"uckzuf"uhren. Das Monitoring Tool Zabbix hat dazu selber %
eine Warnung ausgegeben. Diese besagt, dass der freie Festplattenspeicher zu mehr als 80 \% %
belegt ist. Dies gibt einen ersten Hinweis darauf, dass die 2000 Megabyte Dateien an %
diesen Host nicht richtig gesendet werden k"onnen.  

\begin{figure}[htbp]
\centering
\includegraphics*[width=1\linewidth]{Abb/Tusk20prozent}

\caption{Fehlermeldung bez"uglich der Festplatte von Tusk}
\label{fig:Eth0DazzleStandard}
\end{figure}

Schaut man in die Logfiles und betrachtet die Zeit die zwischen dem Versenden von zwei Paketen vergeht sieht %
man, dass die Zeit die bis zum versenden des n"achsten Packetes %
sehr hoch ist in diesem Fall 17 Minuten. 

\begin{verbatim}
[Thu 16 Jun 19:36:11 CEST 2016]TRANSPORT FAILED 
[Thu 16 Jun 19:53:53 CEST 2016]TRANSPORT SUCCESSFUL
\end{verbatim} 

Das "Ubertragen von erfolgreichen Paketen dauert nach der \cref{tab:compZeiten} durchschnittlich 1534,23 Sekunden %
also ungef"ahr 25 Minuten. Mit diesem Wissen kann man die Schlussfolgerung ziehen, dass der Abbruch des verschickten %
Paketes an den Host Tusk erst sehr sp"at in der "Ubertragung passiert. SCP "uberpr"uft vor dem Versenden einer Datei nicht %
nach, ob genug Speicherplatz f"ur diese Datei vorhanden ist. So kommt es auch, dass der Host Tusk auch im %
Test auf dem Ethernet Port eine eingehende Last aufweisen kann. Da SCP auf der SSH aufbaut welches das %
TCP Protokoll verwendet, werden die verschickten Dateien segmentiert. F"ur gew"ohnlich sind solche Segmente %
1500 Byte gro"s. Dies bedeutet, dass der Host Tusk solange 1500 Byte TCP-Segmente empf"angt, bis bei diesem %
die Festplatte voll ist und keine weiteren Segmente mehr annehmen kann. Daraus folgt dann, dass die Daten"ubertragung %
abgebrochen wird, Zabbix hat f"ur die verschickten Segmente Statistiken sammeln k"onnen, wie z.B: Den eingehenden %
Traffic und Belastung der CPU, obwohl der Host nie die komplette Datei empfangen hat. % 

\section{Schlussfolgerung}
\label{sec:schluss}

Wie bei vielem in der Informatik ist es nicht m"oglich \emph{das Beste} zu definieren %
und danach eine Auswahl zu treffen. Es muss st"andig abgewogen werden was in einem %
Use Case am wichtigsten ist. Jeder der gezeigten Tests, hat den anderen %
Tests gegen"uber Vor- und Nachteile. Deshalb betone ich an dieser Stelle nochmal die %
Zielsetzung die in \cref{cha:einfuehrung} definiert wurde. 

\begin{quotation}
Das Ziel dieser Bachelorarbeit ist es herauszufinden, ob "uber die Paketgr"o"se eine %
Optimierung der Performanz der Endger"ate und der Daten"ubertragung im Netzwerk m"oglich ist. %
Dabei wird versucht die gr"o"stm"ogliche Menge an Daten im Netzwerk auszutauschen ohne die %
Funktionalit"at eines der Hosts zu gef"ahrden. %
\end{quotation}

Wie man in \cref{cha:versuche} sehen konnte, gibt es signifikante Unterschiede bei den in dieser %
Ausarbeitung durchgef"uhrten Testf"alle. Deshalb wird der Fokus erstmal darauf liegen, die %
Anforderung der Zielsetzung zu erf"ullen. Mit dem Abschnitt der Zielsetzungi, das die %
Funktionalit"at der Hosts nicht gef"ahrdet wird, f"allt der 2000 Megabyte Testversuch %
schonmal als eine geeignete Paketgr"o"se raus. Aufgrund der in \cref{sec:fehler} % 
hohen Paketgr"o"se kann der Host Tusk nicht fehlerfrei arbeiten. Seine CPU %
ist zu 88 \% ausgelastet und Tusk kann keine empfangen. Dies alleine ist schon %
ein Kriterium, dass 2000 Megabyte Pakete ausschlie"st. Dass auch noch 35,16 Gigabyte %
an Daten im Netz verloren gegangen und insgesamt nur 154,30 Gigabyte verschickt wurden, %
erf"ullt auch die zweite Anforderung nicht, n"amlich m"oglichst viele Daten zu verschicken. %
Damit f"allt die m"ogliche Auswahl auf die 20 Megabyte Dateien oder die 200 Megabyte %
Dateien. Dazu wird als erstes die Menge der verschickten Gigabyte miteianander verglichen. %
Aus der \cref{tab:compPackages} l"asst sich ablesen, dass bei 200 Megabyte 224,80 %
Gigabyte an Daten verschickt worden sind und keine einziges Packet verloren gegangen ist. %
Beim 20 Megabyte Test wurden hingegen nur 213,60 Gigabyte verschickt. Dabei sind 80 Megabyte %
an Daten verloren gegangeni, was viel weniger als 0,01 \% ist. Deshalb sind 200 Megabyte Testdateien %
f"urs erste am geeigneitsten. Jedoch muss noch der zweite Punkt beachtet werden, n"amlich % 
dass die Hosts im System nicht komplett "uberlastet werden und noch simple Aufgaben %
erf"ullen k"onnen. Dazu werden die Werte aus der \cref{tab:compCPU} miteinanander verglichen. %
Der Leerlaufprozess beim 20 Megabyte Test betr"agt 23,83 \% und beim 200 Megabyte Test \% 23,35 \%. %
Obwohl das hei"st das, dass die Hosts im Test mit 20 Megabyte Dateien weniger belastet waren als im %
Test mit 200 Megabyte Dateien, ist der Unterschied sehr geringf"ugig. Deshalb h"alt man sich %
bei der Entscheidung, welche der Dateigr"o"sen am besten geeignet ist daran, welche der beiden %
Testf"alle die meisten Daten verschickt hat, welche der 200 Megabyte Test war. %

Da aber, wie am Anfang dieses Abschnittes erw"ahnt, die optimale Dateigr"o"se %
vom jeweiligen Use Case abh"angig ist muss man von Use Case zu Use Case unterscheiden. %
Als Beispiel in dem die 20 Megabyte Datei von Vorteil w"are, ist in eine Netzwerkanwendung %
wo die Zeit ein wichtige Rolle spielt und es nicht auf Menge der Daten ankommt sondern %
das schnell Daten ankommen die 20 Megabyte Datei gr"o"se die bessere Wahl. %


