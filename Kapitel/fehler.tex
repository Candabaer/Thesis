\section{Aufgetretene Fehler}
\label{sec:fehler}

Wie schon in \cref{sec:2000MBTest} erw"ahnt kam es bei diesem Test zu einem Fehler %
bei einem der Hostsysteme. Alle Pakete, die zum Host verschickt werden sollten, %
haben ihr Ziel nicht erreicht. Dadurch sind sehr viele Pakete verloren gegangen. %
Insgesamt konnte vom Host Tusk kein einziges Paket angenommen werden. So sind alle %
fehlerhaften Pakete auf ihn zur"uckzuf"uhren. Das Monitoring Tool Zabbix hat dazu selber %
eine Warnung ausgegeben. Diese besagt, dass der freie Festplattenspeicher zu mehr als 80 \% %
belegt ist. Dies gibt einen ersten Hinweis darauf, dass die 2000 Megabyte Dateien an %
diesen Host nicht richtig gesendet werden k"onnen.  

\begin{figure}[htbp]
\centering
\includegraphics*[width=1\linewidth]{Abb/Tusk20prozent}

\caption{Fehlermeldung bez"uglich der Festplatte von Tusk}
\label{fig:Eth0DazzleStandard}
\end{figure}

Schaut man in die Logfiles und betrachtet die Zeit die zwischen dem Versenden von zwei Paketen vergeht sieht %
man, dass die Zeit die bis zum versenden des n"achsten Packetes %
sehr hoch ist in diesem Fall 17 Minuten. 

\begin{verbatim}
[Thu 16 Jun 19:36:11 CEST 2016]TRANSPORT FAILED 
[Thu 16 Jun 19:53:53 CEST 2016]TRANSPORT SUCCESSFUL
\end{verbatim} 

Das "Ubertragen von erfolgreichen Paketen dauert nach der \cref{tab:compZeiten} durchschnittlich 1534,23 Sekunden %
also ungef"ahr 25 Minuten. Mit diesem Wissen kann man die Schlussfolgerung ziehen, dass der Abbruch des verschickten %
Paketes an den Host Tusk erst sehr sp"at in der "Ubertragung passiert. SCP "uberpr"uft vor dem Versenden einer Datei nicht %
nach, ob genug Speicherplatz f"ur diese Datei vorhanden ist. So kommt es auch, dass der Host Tusk auch im %
Test auf dem Ethernet Port eine eingehende Last aufweisen kann. Da SCP auf der SSH aufbaut welches das %
TCP Protokoll verwendet, werden die verschickten Dateien segmentiert. F"ur gew"ohnlich sind solche Segmente %
1500 Byte gro"s. Dies bedeutet, dass der Host Tusk solange 1500 Byte TCP-Segmente empf"angt, bis bei diesem %
die Festplatte voll ist und keine weiteren Segmente mehr annehmen kann. Daraus folgt dann, dass die Daten"ubertragung %
abgebrochen wird, Zabbix hat f"ur die verschickten Segmente Statistiken sammeln k"onnen, wie z.B: Den eingehenden %
Traffic und Belastung der CPU, obwohl der Host nie die komplette Datei empfangen hat. % 
