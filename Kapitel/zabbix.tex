\chapter{Zabbix}
\label{cha:zabbix}
In \cref{cha:framework} wurde schon die verwendete Netzwerk Monitoring Software angeschnitten. In diesem Abschnitt wird nun %
ein tieferer Einblick in den Aufbau und die Verwendung von Zabbix gew"ahrt. Zabbix erm"oglicht es %
auch das Monitoring als ein verteiltes System mit mehreren Servern aufzubauen. Diese sogenannten Proxys %
sind jedoch kein Bestandteil dieser Thesis und werden deshalb nicht weiter betrachtet.

\section{Zabbix Server}
\label{sec:server}
Der Zabbix Server ist die zentrale Komponente des vorgestellten Frameworks. Dem Server werden von den Agents und Proxys %
Informationen zugeschickt. Die Aufgabe des Servers ist es, die Daten zu speichern und zu verarbeiten. %
Der Server speichert auch die Konfiguration der einzelnen Agents, die sich im Netzwerk befinden. %
Um die Konfiguration der Agents einstellen zu k"onnen, m"ussen jedoch erst einmal die Agents im Zabbix %
Server registriert werden. Die dazu ben"otigten Daten sind die Agent IP und ein eindeutiger Name. %
Wenn nun der Agent im Server registriert worden ist, kann man ihn im Zabbix Server verschiedene Templates auflegen. %
Das in dem in \cref{cha:framework} gezeigte Framework verwendet prim"ar das Template \emph{Template OS %
Linux}. Dieses Template ist eines von vielen vordefinierten Templates. Es ist auch m"oglich selber Templates %
zu erstellen. Templates sind in einem XML-Format abgespeichert und k"onnen dadurch einfach unter Nutzern von Zabbix % 
ausgetauscht werden. Die Firma, die den Vertrieb von Zabbix regelt, hat extra daf"ur die Zabbix Share eingef"uhrt %
auf dem Zabbix Templates und vieles mehr ausgetauscht werden kann \autocite{zabbix:share}. \ \\ %
Templates enthalten sogenannte Items, die "uber Active Checks oder Passive Checks Informationen von %
einem vorher im Zabbix Server registrierten Agent anfordern. Ein Passive Check geschieht "uber einen %
Request. Diese sind in JSON verfasst und werden geb"undelt verschickt um Bandbreite zu sparen. %
Der Prozess beeinhaltet f"unf Schritte: %
\begin{enumerate}
\item Der Server "offnet eine TCP Verbindung. %
\item Der Server sendet einen Request. Als Beispiel agent.version. %
\item Der Agent empf"angt den Request und antwortet mit der Versionsnummer. %
\item Der Server verarbeitet die Antwort und erh"alt als Ergebniss Zabbix3.0.0rc. %
\item Die TCP Verbindung wird geschlossen. %
\end{enumerate}
\autocite{zabbix:activepassive}. Passive Checks und Active Checks unterscheiden sich darin, wer den Request %
ausl"ost. Bei einem Passiv Check sind die Hosts selber inaktiv bis vom Server ein Request eintrifft. Bei Active %
Checks wird der Agent selber aktiv. Der Agent sendet dann einen Request an den Server, in dem der Agent nach den %
Daten, fragt die der Server erwartet \autocite{zabbix:activepassive2}. %
Dies ist der Fall, wenn vom Agent Daten abgefragt werden sollen, die nicht vom %
Betriebssystem des Agents gesammelt werden. Das bedeutet, dass der Agent selber nun die Daten, die der Server erwartet, %
sammeln muss. Dies geschieht meistens "uber externe Programme. Sollte ein Agent zu viele Active Checks haben, kann es %
sich auf die Performanz des Hosts auswirken. \ \\ % 

"Uber die API ist es m"oglich Programme f"ur den Server zu schreiben, mit denen man zum Beispiel %
eine eigene Benutzeroberfl"ache erstellen kann um die Konfigurationen auf dem Server zu verwalten. %
"Uber die API ist es auch m"oglich einen eigenen Zugriff auf die dem Server zugrunde liegende %
Datenbank herzustellen. %
Der Zabbix Server basiert auf einem Apache Web Server. Um das Zabbix Frontend nutzen zu k"onnen, wird %
PHP 5.4.0 oder aufw"arts ben"otigt. PHP 7 wird jedoch noch nicht unterst"utzt. Die Daten und Statistiken, die Zabbix sammelt, %
werden in einer Datenbank gespeichert. Es werden f"unf verschiedene Datenbanken unterst"utzt. %
\begin{itemize}
\item MySql Version 5.0.3 aufw"arts.
\item Oracle Version 10g aufw"arts.
\item PostgreSQL Version 8.1 aufw"arts.
\item SQLite Version 3.3.5 aufw"arts.
\item IBM DB2 Version 9.7 aufw"arts (Noch nicht fehlerfrei).
\end{itemize}
\autocite{zabbix:req}.
Der Zabbix Server selber ist auch als ein Agent konfiguriert, der sich selber "uberwacht. Der Server %
ist jedoch nicht im allgemeinen Prozess des in \cref{cha:framework} vorgestellten Frameworks t"atig. %
Der Server gibt aufschluss "uber die sogenannte \emph{Zabbix Server Perfomance} diese betrachtet zwei Dinge, einmal die sogenannte Queue, welche %
angibt wie viele von den Agents "ubermittelten Werten darauf warten vom Zabbix Server verarbeitet zu werden und %
die verarbeiteten Werte pro Sekunde. Diese Werte lassen einen Schluss auf die Leistungsanforderungen des Servers zu. %
Sollte die Queue einen bestimmten Schwellenwert erreichen, wird auf dem Zabbix Frontend eine Warnung ausgegeben, dass der %
Server mit der Verarbeitung nicht hinterherkommt. Dies kann so weit gehen, dass die gesammelten Daten auf dem Server fehlerhaft %
sind. Jedoch ist dieses Szenario in diesem Framework unwahrscheinlich, da die Gr"o"se des %
Frameworks "uberschaubar ist. Der Traffic auf dem Ethernet Port Eth0 wird "uberwacht, da alle von den Agents gesammelten Informationen %
"uber diesen an den Server gelangen. Dabei wird der eingehende sowie auch der ausgehende Traffic betrachtet.

\section{Zabbix Agent}
\label{sec:agent}
Der Zabbix Agent wird auf dem zu "uberwachenden System installiert. Der Agent sammelt die Daten "uber die im Betriebssystem %
integrierte Monitoring Funktion oder "uber die Zabbix eigene API und sendet diese Informationen je nach Art des Checks an den Agent. %
Da der Agent schon im Betriebssystem integrierte Funktionen, benutzt ist dieser sehr effizient und verbraucht kaum %
Ressourcen des Host Systems. Anders als der Server besitzt der Agent kein eigenes Frontend und speichert die Werte nicht %
in einer Datenbank. Der Agent ist so konstruiert, dass dieser weitestgehend ohne Administratorrechte funktionieren kann. % 
Da es auch m"oglich ist Agent Hostsysteme "uber einen Fernzugriff auszuschalten oder neuzustarten, m"ussen dem Agent f"ur diese Funktionen %
die Rechte zugeteilt werden. Dadurch wird das Sicherheits %
risiko f"ur den Host erheblich reduziert. F"ur gew"ohnlich wird f"ur den Agent ein eigener User angelegt. Die Anwendung des Agents %
l"auft unter Unix Systemen als ein Daemon und unter Windows als ein Systemdienst. Zabbix unterst"utzt jedoch noch mehr %
Betriebssysteme. Eine Auswahl davon sind: 
\begin{itemize}
\item Linux.
\item Windows: alle Desktop und Server Versionen seit 2000.
\item OpenBSD.
\item Mac OS X.
\item Solaris 9,10,11.
\end{itemize}
\autocite{zabbix:agent}. %
Im Aufbau des Frameworks, das in dieser Arbeit weiter betrachtet wird, laufen die Zabbix Agents unter dem g"angigen Raspberry Pi %
Betriebssystem Raspbian, welches ein Debian Port ist. Um den Host jedoch verwenden zu k"onnen, muss man in %
den Konfigurationsdateien die Ports, die der Server f"ur das Versenden von Requests benutzt, eingestellt werden. Auch die IP des Servers %
muss eingetragen werden, da der Agent sonst eingehende Requests verwirft um so die Sicherheit f"ur den Host zu gew"ahrleisten. %
In den Konfigurationsdateien der Agents ist es auch m"oglich %
die Active Checks zu notieren, was in diesem Versuchsaufbau auch gemacht wurde. Das Template OS Linux hatte keine Items hatte die das %
"uberwachen von der Festplattenauslastung gew"ahrleistet. Dies kann man "uber die Konfigurationsdateien einstellen: %
\begin{quote}

UserParameter=custom.vfs.dev.read.ops[*],customParaScript.sh \$1 4 

\end{quote}

Wenn dann in der Antwort zu einem Active Check nach dem \emph{custom.vfs.dev.read.ops[*]} gefragt wird, wird das Bash Script customParaScript.sh %
ausgef"uhrt. Die Auszeichnung \emph{UserParameter=} gibt an, dass es sich hierbei um ein selbstdefiniertes Item handelt, welches mithilfe von Zabbix %
"uberwacht werden soll. Das Script customParaScript.sh dient dazu die Last auf der Festplatte zu "uberpr"ufen. %
Dazu liest es aus der Datei die in Unix Systemen unter /proc/diskstats liegt. Die den zugeh"origen Werten \$1 und 4 sind Variablen, die f"ur die %
Verarbeitung notwendig sind. Der in diesem Beispiel genannte User Parameter liest die Lese-Operationen pro Sekunde aus der Datei und leitet diese %
an den Server weiter.

