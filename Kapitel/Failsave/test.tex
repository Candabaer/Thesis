\chapter{Testaufbau}
\label{cha:testaufbau}
Mit dem in \cref{cha:dasframework} vorgestellten Testframework, wird die Netzwerk Perfomance gepr"uft. %
Dabei wird vorallem ein Hauptaugenmerk auf die CPU Last, die Festplatten Last und die Last auf den %
Ethernet Ports der Raspberry Pis gelegt. Ausserdem wird die Zeit gemessen die benötigt ist um %
die Daten von einem Paket zum anderen zu verschicken und die Anzahl der gescheiterten Paket"ubertragungen. %
In diesen Tests wird versucht eine optimale Paketgr"o"se f"ur die Daten"ubertragung in einem %
Local Area Network zu finden in denen Linux Endger"ate aufgestellt sind. Daf"ur werden die Tools verwendet, %
die im \cref{cha:dasframework} vorgestellt worden sind.

\section{Testbeschreibung}
\label{sec:testbeschreibung}
In diesem Abschnitt sind die Tests beschrieben die im Rahmen der Arbeit durchgef"uhrt worden sind. %
Alle Tests sind f"ur eine Dauer von f"unf Stunden aufwärts ausgelegt. Die Tests werden in einem %
St"orungsfreien Netzwerk gemacht. Dabei werden Pakete von drei verschiedenen gr"o"sen miteinander %
verglichen dazu geh"oren:  
\begin{description}
\item[20 Mega Byte Pakete: ]In diesem Test werden 20 Mega Byte gro"se Pakete verschickt, als Ergebniss %
wird erwartet das in diesem Test die meisten Pakete verschickt werden und die geringste Fehlerrate an %
gescheiterten Paketen zu erwarten ist. 
\item[200 Mega Byte Pakete: ]Hier werden Pakete mit einer gr"o"se von 200 Mega Byte verschickt. Es %
ist zu erwarten das einige Pakete ihr Ziel nicht erreichen werden. 
\item[2 Giga Byte Pakete: ]
\end{description}
