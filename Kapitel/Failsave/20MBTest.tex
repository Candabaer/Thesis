\subsection{20 Megabyte Testlauf}
\label{subsec:20mbTestlauf}
Der erste durchgef"uhrte Test basiert auf 20 Megabyte Paketen, dazu wurde das Hintergrundrauschen %
so eingestellt das die gr"o"se der verschickten Pakete 20 Megabyte betr"gt. 

\begin{figure}[htbp]
\centering
\includegraphics*[width=0.9\linewidth]{Abb/ZabbixDazzle/Standard/DazzleStandard}

\caption{Traffic auf Eth0 bei Normalbetrieb auf Dazzle}
\label{fig:Eth0DazzleStandard}
\end{figure}

Wie man in \cref{fig:Eth0DazzleStandard} sehen kann ist der durchschnittlich eingehende Durchsatz auf dem Pi Dazzle 8,04 Megabit %
pro Sekunde. Der ausgehende Traffic betr"agt durchschnittlich 8,62 Megabit pro Sekunde. Rechnet man diese Werte in Bytes pro Sekunde um %
betr"agt der ausgehende Datenstrom 1,0775 Megabyte/s und der durchschnittlich eingehende Datenstrom 1,005 Megabyte/s %
Daraus k"onnen wir schlie"sen das der Ethernet Port von Dazzle unter einer durschnittlichen I/O-Last von 2,0825 Megabyte/s stand. % 
Woraus folgt das der Ethernet Port zu 20,825\% belastet war.
\begin{table}
\centering
\begin{tabular}{l%
 r<{\,Mb/s}%
 r<{\,Mb/s}%
 r<{\,Mb/s}%
 r<{\,\%}%
}
Agent  				& Eingehende		& Ausgehende		& Gesamt		& Auslastung von Eth0	\\
\hline
DotA				& 0,0769		& 0,1222		& 0,19900		& 0,0199 		\\
Dazzle 				& 8,04 			& 8,62			& 16,66 		& 1,666			\\
Tusk 				& 8,48			& 8,37			& 16,85			& 1,685			\\
Tinker				& 8,42			& 8,56			& 16,98			& 1,698			\\
Lion				& 8,42			& 8,65			& 16,89			& 1,707			\\ 
$\diameter $ Agent 		& 8,34			& 8,55 			& 16,89			& 1,689 		\\   
$\diameter $ Agent \& Server 	& 6,68736		& 6,86444		& 13,5518		& 1,35518		\\

\end{tabular}
\caption{Normalbetrieb Traffic auf allen Pis}
\label{tab:standardTraffic}
\end{table}

In der \cref{tab:standardTraffic} sind die Durchnittswerte f"ur den Traffic der jeweiligen Agents eingespeichert. Auch der %
des Servers, da dieser im selben Netzwerk aufgestellt ist wie die anderen Agents. Die Tabelle zeigt uns auch %
das sich die Agents alle in einem "ahnlichen Umfeld befinden was deren Input sowie Output betrifft. So betr"agt die
Standardabweichung der Pis nach der \cref{tab:standardTrafficAbweichung}     

\begin{table}
\centering
\begin{tabular}{l%
 r<{\,Mb/s}%
 r<{\,Mb/s}%
 r<{\,Mb/s}%
 r<{\,\%}%
}
Agent		& Eingehende            & Ausgehende            & Gesamt                & Last auf Eth0 \\
\hline
Agents		& 0,174929              & 0,108857              & 0,15411               &  0,01541        \\
Agents \& Server& 3,308981		& 3,372526        	& 6,67782             	&  0,66782         \\
\end{tabular}
\caption{Normalbetrieb Standarbweichung der Werte}
\label{tab:standardTrafficAbweichung}
\end{table}

\begin{table}
\centering
\begin{tabular}{l%
 r<{\,KB/s}%
 r<{\,Ops/s}%
}
Agent	  			& Schreiben	 	& Input/Output 	\\	
\hline
Dazzle 				& 1,04			& 154,09	        \\
Tusk 				& 1,11			& 42,4			\\
Tinker				& 1,07 			& 142,5		 	\\
Lion				& 1,04			& 144,31	 	\\
$\diameter $ Agent 		& 1,065			& 120,825		\\   
Standardabweichung 		& 0,028722813  		& 45,4928		\\
\end{tabular}
\caption{I/O Zeiten bei Normalbetrieb auf den Pis}
\label{tab:NormalbetriebIoStat}
\end{table}


\begin{figure}[htbp]
\centering
\includegraphics*[width=0.9\linewidth]{Abb/ZabbixDazzle/Standard/IoStatDazzleStandard}

\caption{I/O Stats von Dazzle beim Normalbetrieb}
\label{fig:IoStatDazzleStandard}
\end{figure} %

In der \cref{fig:IoStatDazzleStandard} sieht man das die Festplatte des Raspberry Pis konstant beschrieben wird. Im durchschnitt werden 1.04 Kilobytes %
die Sekunde geschrieben. Mit einem einer Maximallast von 1.86 Kilobytes die Sekunde. Was nicht ann"ahrend die Maximale Schreibgeschwindigkeit der verwendeten %
SanDisk SD Karten ist welche bei 30 Megabyte\autocite{san:sd} pro Sekunde liegen. 

\subsubsection{Schlussfolgerung Test: Normalbetrieb}
\label{subsubsec:schlussfolgerung:normalbetrieb}
Nach erfolgreichem durchf"uhren des Tests kann man sehen wie sich die einzelnen Endger"ate im Netzwerk verhalten wenn Pakete von 20 MB gr"o"se dar"uber l"auft. %
Man kann sehen das der Datenstrom konstant bleibt und es keine hohen Abweichungen im eingehenden und ausgehenden Traffic gibt. Woraus man schlie"sen kann das %
ein Reibungsloser Ablauf im Netzwerk geleistet ist. Jedoch sieht man das Tusk sehr wenige I/O Operationen pro Sekunde vornimmt. Um das genauer beurteilen zu k"onnen %
m"usste man sich angucken wie die CPU Zeiten genutzt werden. Jedoch ist zu bedenken das die Metrekt der I/O Operationen pro Sekunde  



