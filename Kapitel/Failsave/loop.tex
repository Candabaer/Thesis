\subsection{Loop}

\label{subsec:loop}
Wie in \cref{sec:testbeschreibung} beschrieben ist ein Loop ein Fehler der auftritt wenn in einem Switch zwei Ports miteinander verbunden worden sind. %
Dasselbe gilt auch wenn zwei Switches miteinander verbunden werden. Loops sind Fehler die auftreten wenn der Endnutzer nicht mit dem Umgang der Hardware vertraut ist, %
oder die IT-Infrastruktur zu gro"s wird. %
Der erwartete Ausgang des Tests ist nach Abschnitt 3.8 \autocite{book:schreiner} ein totaler Ausfall des Netzwerkes. Die nun gezeigten Ergebnisse best"atigen diese %
Annahme.  

\begin{figure}[htbp]
\centering
\includegraphics*[width=0.9\linewidth]{Abb/ZabbixDazzle/Loop/DazzleLoop}

\caption{Traffic auf Eth0 bei einem Loop}
\label{fig:Eth0DazzleLoop}
\end{figure}

Wie man in der Graphik sehen kann bricht die Verbindung zum Zabbix Server komplett ab. Es findet nichtmal der wie in \cref{subsec:nix} aufgezeigte Server Poll statt, % 
der einen geringen Traffic erzeugt. %
Auch der Graph f"ur die Festplatten Last bricht zum Zeitpunkt des Loops ab. Wieder werden im Graphen auch die Durchschnittswerte des Traffics aufgezeigt. %
Wenn man nun die Werte mit denen aus dem \cref{subsec:normalbetrieb} vergleicht stellt man fest das die Durchschnittswerte eine geringe Abweichung aufzeigen. %
Dies liegt dem zugrunde das Zabbix keine neue Daten von den Agents erhalten kann. Da die Berechnung des durchschnittlichen Traffic erst dann erfolgt wenn %
der Agent eine Naricht "uber seinen Zustand versendet. Da diese jedoch nicht den Server erreichen bleiben die Durchschnnittswerte wie beim Normalbetrieb. %

\begin{figure}[htbp]
\centering
\includegraphics*[width=0.9\linewidth]{Abb/ZabbixDazzle/Loop/IoStatDazzleLoop}

\caption{I/O-Last auf der Festplatte bei einem Loop}
\label{fig:IoStatDazzleLoop}
\end{figure}

Hier kann man sehen das der Loop auch die Informationen die der Agent "uber die Festplatte verschickt unerreichbar sind. Dies hat jedoch aus Erfahrungswerten keinen Einflu"s auf die %
Performanz der Agents. Man kann davon ausgehen das sich die I/O-Last der Agents auf den Wert der in \cref{subsec:nix} gezeigten Werte einpendelt. %


\subsubsection{Schlussfolgerung Test: Loop}
\label{subsubsec:schlussfolgerung:loop}

Wie man in \cref{subsec:loop} sehen kann, zeichnet sich ein Loop dadurch aus der komplette Traffic in einem Netzwerk zusammenbricht. In Zabbix selber zeichnet sich das %
dadurch aus das die Graphen einen direkten Schnitt aufzeigen und auch "uber einen l"angeren Zeitraum sich die Durchschnittswerte die von den Agents verschickt werden nicht %
ver"andern. Auch die Logfiles geben aufschluss "uber die Transportierten Pakete so kann man aus diesen lesen das nur noch die Pakete die an den Lokalen Host geschickt werden %
ihr Ziel erreichen.  



