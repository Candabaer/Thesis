\pagenumbering{arabic}
\chapter{Einf"uhrung}
\label{cha:einfuehrung}

\section{Einf"uhrung in das Thema} 
Computernetzwerke sind inzwischen fester Bestandteil der Gesellschaft und finden "uberall Verwendung. %
Wenn fr"uher Netzwerke nur im Milit"ar, in Universit"aten und Firmen benutzt wurden, haben nun auch %
normale Endverbraucher Zugriff auf Netzwerke. Inzwischen haben alle die M"oglichkeit auf das Internet, welches "uber den %
gesamten Globus verteilt ist, zuzugreifen. Jedoch ist auch das Internet nur ein Netzwerk aus Netzwerken. %
Geht man in die unterste Ebene kommt man irgendwann in einem sogenannten Local Area Network (LAN) an. Inzwischen sind diese in fast jedem %
Haushalt zu finden, da sie es erm"oglichen mehrere Computer "uber einen Router mit dem Internet %
zu verbinden. LANs erm"oglichen einen Datenaustausch zwischen den im Netzwerk verbundenen Computern. % 
Ein sehr beliebter Verwendungszweck sind die sogenannten LAN-Partys, bei denen "uber %
ein LAN Videospiele miteinander gespielt werden k"onnen. %
Dadurch ist es m"oglich lokale verteilte Systeme aufzubauen, die gemeinsam eine Aufgabe %
bearbeiten. Das Berkeley Open Infrastructure for Network Computing stellt eine Anwendung %
f"ur verteilte Systeme dar \autocite{uni:boinc}. Mit dieser ist es m"oglich "uber das %
Internet an verschiedenen Forschungsprojekten teilzunehmen, indem man eigene %
Rechenzeit zur Verf"ugung stellt. Eine weitere Anwendung f"ur verteilte Systeme stellt %
das Zabbix Framework dar, welches auch in dieser Bachelor Arbeit n"aher betrachtet wird. %
Zabbix ist ein Netzwerk Monitoring Tool, mit dem es m"oglich ist St"orungen in einem  Netzwerk %
zu erkennen und die Leistung von den im Netzwerk angeschlossenen Computern zu betrachten. % 

%Motiviation und Zielsetzung zusammenfügen. 
\section{Motivation der Bestimmung einer optimalen Paketgr"o"se}
In einem Netzwerk, in dem konstant Daten ausgetauscht werden, wirkt sich die Gr"o"se der %
versandten Pakete auf die Leistung der Endger"ate aus. Um eine gleichm"a"sige Auslastung des Netzwerkes %
und die Leistungsf"ahigkeit der im Netzwerk angeschlossenen Computer zu gew"ahrleisten, %
kann man dies mittels der Gr"o"se der Pakete steuern. Deshalb ist das bestimmen einer Paketgr"o"se %
f"ur ein gegebenes System wichtig, um eine gleichm"a"sige Lastenverteilung im Netzwerk zu haben %
und die Endger"ate nicht zum abst"urzen zu bringen. % 

\section{Zielsetzung dieser Arbeit}
Das Ziel dieser Bachelorarbeit ist es herauszufinden, ob "uber die Paketgr"o"se eine %
Optimierung der Performanz der Endger"ate und der Daten"ubertragung im Netzwerk m"oglich %
ist.  


\section{Methoden zum Erreichen des Ziels}

Zabbix, welches von der Zabbix SIA vertrieben wird, ist wie Nagios eine Open Source % Nagios drinalssen?
Netzwerkmonitoring Software, die es erm"oglicht Ger"ate in einem Netzwerk zu "uberwachen. %
Daf"ur wurde ein Local Area Network bestehend aus vier Raspberry Pis aufgebaut. Das Netzwerk wurde %
in einer Sterntopologie aufgebaut, welche standardm"a"sig verwendet wird. Der Vorteil bei dieser Topologie ist, dass %
der Ausfall von einem Computer keine Auswirkung auf den Rest des Netzwerks hat, sie leicht erweiterbar ist und %
hohe "Ubertragungsraten bietet, wenn der Netzknoten ein Switch ist \autocite{book:CN2003}. % 
Daf"ur wird das auf dem TCP/IP-Protokollstapel basierende Secure Copy Programm verwendet. %
Es stellt eine Erweiterung der Unix Secure Shell dar und erm"oglicht einen fehlerfreien %
Austausch von Dateien. Beide Programme, Secure Copy und Secure Shell sind inzwischen fester Bestandteil %
fast aller Linux Distributionen, so auch dem Debian Port Raspbian, welcher auf den %
Raspberry Pis installiert ist. Mit diesen Tools wurde im Rahmen dieser Bachelorarbeit ein eigenes Testframework aufgebaut, %
welches automatisiert Dateien von den Computern verschickt und Informationen "uber die Dateien, %
Empf"anger und "Ubermittlungszeiten speichert. Diese Logfiles werden %
mittels Textverarbeitung analysiert und in Tabellen dargestellt. Zabbix kann %
Statistiken zu den Computern im Netz erstellen. So ist es m"oglich mittels Zabbix die %
CPU-Last, die Festplattenauslastung und die Last auf den Ethernet Ports der Raspberry Pis %
zu beobachten. Dieses Framework wird in \cref{cha:framework} nochmal genauer vorgestellt. %
Au"serdem werden in dieser Arbeit % Marie oder vllt. weglassen?!
drei verschiedene Testf"alle betrachtet, kleine Dateien 20 Megabyte, %
mittlere Dateien 200 Megabyte und gro"se Dateien mit 2000 Megabyte. %
Zum Schluss werden die zuvor gesammelten Ergebnisse in \cref{cha:ergebnisse} verglichen und die %
Annahme "uberpr"uft, ob es m"oglich ist eine optimale gr"o"se f"ur Pakete in einem Netzwerk %
zu bestimmen. %







